\documentclass[a4paper, 12pt, oneside]{article}
\usepackage[OT1]{fontenc}
\usepackage{booktabs}
\setlength{\emergencystretch}{15pt}
\usepackage{fancyhdr}
\begin{document}
\begin{titlepage} % Suppresses headers and footers on the title page
	\centering % Centre everything on the title page
	\scshape % Use small caps for all text on the title page

	%------------------------------------------------
	%	Title
	%------------------------------------------------
	
	\rule{\textwidth}{1.6pt}\vspace*{-\baselineskip}\vspace*{2pt} % Thick horizontal rule
	\rule{\textwidth}{0.4pt} % Thin horizontal rule
	
	\vspace{1\baselineskip} % Whitespace above the title
	
	{\LARGE Corals in the Meteorites} % Title
	
	\vspace{0.5\baselineskip} % Whitespace below the title
	
	\rule{\textwidth}{0.4pt}\vspace*{-\baselineskip}\vspace{3.2pt} % Thin horizontal rule
	\rule{\textwidth}{1.6pt} % Thick horizontal rule
	
	\vspace{1\baselineskip} % Whitespace after the title block
	
	%------------------------------------------------
	%	Subtitle
	%------------------------------------------------
	
	{Dr. David Friedrich Weinland} % Subtitle or further description
	
	\vspace*{1\baselineskip} % Whitespace under the subtitle
	
    {\small Das Ausland, No. 16, Article 1} % Subtitle or further description
    
	%------------------------------------------------
	%	Editor(s)
	%------------------------------------------------
	
	\vspace{1\baselineskip}

	{\small\scshape Stuttgart --- April 17, 1881 \\ }
			
    \vspace*{\fill}

    Internet Archive Online Edition  % Publication year
	
	{\small Attribution NonCommercial ShareAlike 4.0 International } % Publisher
\end{titlepage}
\setlength{\parskip}{1mm plus1mm minus1mm}
\clearpage
\paragraph{}
The question of whether or not celestial bodies besides our Earth are inhabited or were inhabited by living beings is certainly one of the most interesting that exists for the thinking human being and could be, in all probability, already confirmed. The quite analogous physical conditions, as demonstrated by some of the other planets in our Solar System, and, as they probably represent the countless planets of other star systems, suggests with some certainty that not on Earth alone has higher organic processes of development taken place. But this has always been only a speculation, a hypothesis, however well supported.

But it seems that we have now received a very direct answer to this question and that we can see the real remnants of living beings from another celestial body with our own eyes.

It will hardly be doubted at present that the meteorites, which from time to time enter the Earth's sphere of influence and fall upon it, do not originate from the Earth. The assumption that they are the remnants of another, shattered planet, seems almost universally accepted.

In the meteorites, especially in that class called the chondrites, because of their peculiar round inclusions, our compatriot Dr. Hahn believes to have detected a whole series of organic forms — in thin sections that he has made from these meteorites. Dr. Hahn has recently published a work in which he gives, in thirty-two tables, photographic representations of over a hundred thin sections of meteorites produced mechanically, without the consent of a draftsman, all of which contain various forms that Dr. Hahn decidedly declares as not mineral but organic, and indeed animal, and which he would like to see partly as sponges, partly as corals, and partly as crinoids.

The author did not allow a detailed zoological interpretation of the forms and their comparison with terrestrial ones.

A large number of these pictures will certainly surprise every zoologist and paleontologist. An eye trained for coral structures will immediately be reminded of well-known coral structures in the pictures of Table 1: Figures 5 and 6, Table 8, and Table 15. Even if only a single one of these forms were safely proven to be organic, the spell would be broken, and one would then be confident in approaching the organic interpretation of the remaining.

Regarding the most striking of the above-mentioned forms, most of which are from the famous colossal meteorite of Knyahinya in Hungary (June 9, 1866), let us say a few words.

At our request Dr. Hahn provided the original material itself, including an extremely valuable unique piece, for further investigation and we had full leisure to study these strange pieces with the help of our own rather rich coral collection. The result of this study is the full conviction that, at least in these structures, we are really dealing with the remnants of corals, most of which belong to the Favositidae, a family that has so far only been found as fossils in the Paleozoic, the ancient layers of Earth.

The terrestrial polyp colonies of the \emph{Favosites} are composed of parallel adjacent polyp tubes. From above, where the calyx leads and the living polyps sit, the coral colonies of the \emph{Favosites} show a more or less regular network consisting of the walls of the individual polyps. Moreover, especially characteristic of the \emph{Favosites}, there are found transverse dividing walls in the polyp tubes and fairly regular strings of holes in the walls that serve to establish the vascular connections of the polyp tubes with each other.

Such polyparies, i.e. tube bundles quite \emph{Favosites}-like, occur in a large number of Dr. Hahn's meteorite cuts, which come from various meteorite falls. With full clarity one sees in many of these precisely the same transverse dividing walls with little strings of holes at certain distances from each other, and so often that it is absolutely impossible to think of coincidence here, as if any mineralogist could interpret these little pattern relations, transverse dividing walls, and holes, which are seen at a magnification of two hundred times and could be easily traced up to four hundred eighty times, mineralogically. We are certainly dealing with organic structures, specifically with \emph{Favosites}-like corals.

Unfortunately, most of the cuts go parallel to the tube position of the polyparies, which is due to the fact that Hahn, in order to obtain his cuts, broke up the meteorite masses where the splitting was easiest in accordance with the length of the polyp colony.

Only a single, wonderfully nice cut, the aforementioned unique one from Knyahinya, grants a full view from above through the cup of the polypary and at their stringing together. This preparation alone is certainly conclusive for every coral connoisseur. Unfortunately, the photographic image given by Hahn in his work, Table 10: Figures 3 and 4, does not give the clearest picture, as the object is clearly revealed under a good microscope, since the yellowish coloring of the preparation negatively affected the photography.

This object appears to be a complete, small, rounded coral colony, with its base spread on another coral-like structure. The whole network of calyxes is very clear. The calyxes themselves are dark in the middle, filled with a black mass, while a whitish filling mass surrounds the dark core, and lastly, the walls of each tube always have a sharp line visible at low magnification which at greater magnification divides into two parallel lines so that each polyp tube has its own walls. This network of polyp cups divides linearly and shows further calyxes of different sizes and forms. The latter are found, just as we observe in a lot of corals, especially in the Devonian \emph{Favosites polymorhpus}, to be very irregular and sometimes more defined by curved lines, sometimes by straight restricting lines, large or small, with small calyxes between the larger ones forming a transversal cutting.

All the coral formations in the meteorites are silicified. Magnesium silicates are found, which is why they were interpreted as olivine.

However, there remains a very strange fact about these extraterrestrial coral formations. That is their extraordinary smallness. It is truly a Lilliputian animal world with respect to the terrestrial. The coral colony we have just mentioned, which we will describe and depict in more detail at another time (in honour of the discoverer under the name \emph{Hahnia meteoritica}), is a white dot in the meteorite cut that that is just visible to a good eye. Its largest diameter measures only 0.90 mm, the individual calyxes on average only 0.05 mm. This is the state of affairs: we know of no such terrestrial polyp colony as even calyxes of 1 mm diameter are called very small. But we must be prepared for quite different things in these extraterrestrial organisms. There can easily be forms that we cannot place into our system of zoology, indeed, we are astonished that we can, in these coral formations before us, make such close comparisons with terrestrial ones. This testifies to an extraordinarily similar organic evolution in general or on that planet from which these meteorites originate.

One might still ask how is it possible, that with such a large number of meteorites lying in mineralogical collections and the not insignificant number of researchers dealing with them, that these strange organic formations have only now been discovered. Different circumstances may explain the matter. One: all the meteorites are rare finds and dear pieces, which one does not like to sacrifice, therefore in general only a small number of thin sections are made so that the probability of getting more than just a worthless object is not great. Hahn has produced no less than six hundred cuts with truly extraordinary sacrifices in time and money. Also, the above-mentioned meteorites are usually only examined with a magnifying glass, rarely with strong microscopes, and always with only a few cuts.

Nevertheless, individual observers, especially Director Gümbel in his description of the meteorites of Eichstädt and Schöneberg, probably had such organic forms before them. He describes very well and in detail the columnar fibers, yes, he even speaks of irregularly angular, tiny heaps that arise in cross-sections through these fibers. Here he probably had small \emph{Favosites}-like corals in front of him, but he was not thinking of any organism. But Gümbel does say, as if anticipating the discussion, about the meteorite of Kaba: ``Perhaps it will still be possible to prove the presence of organic beings on extraterrestrial bodies.''

We believe, in accordance with the above, that our tireless compatriot Dr. Hahn has succeeded. If Gümbel had been hit by a chance piece like the above-mentioned unique one, of which there may still be many more in the center of the Knyahinya meteorite mass, he would surely have become the discoverer of this extraordinary fact.

About the sponges and crinoids of Hahn's, perhaps another time!
\rule{\textwidth}{1.6pt}\vspace*{-\baselineskip}\vspace*{2pt} % Thick horizontal rule
\rule{\textwidth}{0.4pt} % Thin horizontal rule
\paragraph{}
Since we wrote the above, Dr. Hahn has given us all the underlying cuts of his meteorite work and additional new ones, all in all over three hundred, for closer zoological examination and investigation. There is a huge abundance of material here, for the majority of the cuts, e.g. those made from Knyahinya, seem to be mostly agglomerated organic debris. Well-preserved forms are, of course, quite rare; it is mostly debris, e.g. quite similar to that observed in young ocean limestone of the Mexican Gulf. After acquiring some practice and comparing many cuts, certain recurring forms can be restored quite easily. Especially developed are the sponges of which I have already determined three specific genera. Of a very characteristic bluish sponge, which often occurs as both young and old specimens, I was able, after some very favorable transverse and longitudinal attacks, to draw the inner structure as clearly as that of a living one. Traces of plants also seem to occur; at least a very striking, arched shield-shaped structure with diameter 0.8 mm, divided by a longitudinal hinge, is most reminiscent of the shield algae, \emph{Cocconeis}. Whether the forms generally addressed by Hahn in his book as crinoids really belong to this class still seems to us to be questionable. Some of them are certainly sponges. — We have not found any trace of higher animal forms, of mollusks, arthropods, etc.; so far, all forms clearly indicate a very young formation of the world body concerned. The entire animal world presented, which certainly belongs to at least fifty different species and which originate from various meteorite cases, even those of the previous century, gives the impression of a coherent creation which undoubtedly stems from a single extraterrestrial world. However, the latest meteorite theory, which derives from the famous Schiaparelli and associates the meteorites with comets and their tails, does not seem to be sustainable according to the above. All these organisms have lived in water, never completely frozen, which we are not able to find in comets. This, too, shows the significance of Hahn's discovery, which will create a zoological foundation that brings us great joy.
\end{document}
