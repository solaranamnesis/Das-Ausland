\documentclass[a4paper, 12pt, oneside]{article}
\usepackage[utf8]{inputenc}
\usepackage[T1]{fontenc}
\usepackage[ngerman]{babel}
\usepackage{fbb} %Derived from Cardo, provides a Bembo-like font family in otf and pfb format plus LaTeX font support files
\usepackage{booktabs}
\setlength{\emergencystretch}{15pt}
\usepackage{fancyhdr}
\begin{document}
\begin{titlepage} % Suppresses headers and footers on the title page
	\centering % Centre everything on the title page
	\scshape % Use small caps for all text on the title page

	%------------------------------------------------
	%	Title
	%------------------------------------------------
	
	\rule{\textwidth}{1.6pt}\vspace*{-\baselineskip}\vspace*{2pt} % Thick horizontal rule
	\rule{\textwidth}{0.4pt} % Thin horizontal rule
	
	\vspace{1\baselineskip} % Whitespace above the title
	
	{\LARGE Korallen in Meteorsteinen} % Title
	
	\vspace{0.5\baselineskip} % Whitespace below the title
	
	\rule{\textwidth}{0.4pt}\vspace*{-\baselineskip}\vspace{3.2pt} % Thin horizontal rule
	\rule{\textwidth}{1.6pt} % Thick horizontal rule
	
	\vspace{1\baselineskip} % Whitespace after the title block
	
	%------------------------------------------------
	%	Subtitle
	%------------------------------------------------
	
	{Dr. David Friedrich Weinland} % Subtitle or further description
	
	\vspace*{1\baselineskip} % Whitespace under the subtitle
	
    {\small Das Ausland, Nr. 16, Artikel 1} % Subtitle or further description
    
	%------------------------------------------------
	%	Editor(s)
	%------------------------------------------------
	
	\vspace{1\baselineskip}

	{\small\scshape Stuttgart --- 17. April. 1881 \\ }
			
    \vspace*{\fill}

    Internet Archive Online Edition  % Publication year
	
	{\small Namensnennung Nicht-kommerziell Weitergabe unter gleichen Bedingungen 4.0 International } % Publisher
\end{titlepage}
\setlength{\parskip}{1mm plus1mm minus1mm}
\clearpage
\paragraph{}
Die Frage, ob au"ser unserer Erde noch andere Himmelsk"orper, zumal Planeten, von lebenden Wesen bewohnt seien oder bewohnt gewesen seien — gewiss eine der interessantesten, die es f"ur den denkenden Menschen gibt, — konnte bisher schon mit gro"ser Wahrscheinlichkeit bejaht werden. Die ganz analogen, physikalischen Verh"altnisse, wie sie einige andere Planeten unseres Sonnensystems nachgewiesenerma"sen darbieten und wie sie wohl unz"ahlige Planeten anderer Sonnensysteme darbieten werden, lie"sen mit einer gewissen Sicherheit darauf schlie"sen, dass nicht auf unserer Erde allein ein h"oherer, ein organischer Entwicklungsprozess werde Platz gegriffen haben. Doch war dies immer nur ein Analogieschluss, eine wenn auch noch so gut gest"utzte Hypothese.

Nun aber scheint es in der Tat, dass wir jetzt eine ganz direkte Antwort auf jene Frage erhalten haben, dass wir wirkliche Reste lebender Wesen von einem anderen Himmelsk"orper mit eigenen Augen sehen k"onnen.

Es wird wohl gegenw"artig kaum mehr bezweifelt werden, dass die Meteorsteine, die von Zeit zu Zeit in die Machtsph"are unserer Erde gelangen und auf sie niederst"urzen, nicht von unserer Erde stammen. Die Annahme, dass sie Reste eines anderen, eines zertr"ummerten Planeten seien, scheint fast allgemein zugegeben.

In solchen Meteoriten nun und zwar besonders in jener Klasse, die man wegen ihrer eigent"umlichen, rundlichen Einschl"usse Chondriten nennt, glaubt unser Landsmann Dr. Hahn eine ganze Reihe organischer Formen — in D"unnschliffen, die er aus jenen Meteorsteinen hergestellt, — nachweisen zu k"onnen. H. Hat dar"uber vor kurzem ein Werk ver"offentlicht, in welchem er auf 32 Tafeln photographische, somit mechanisch, ohne Zu- und Abtun eines Zeichners, hergestellte Darstellungen von "uber hundert D"unnschliffen von Meteoriten gibt, s"amtlich verschiedene Formgebilde enthaltend, welche Dr. Hahn f"ur entschieden nicht mineralische, vielmehr f"ur organische und zwar tierische erkl"art, und die er teils als Schw"amme, teils als Korallen, teils als Krinoiden (Lilienstrahler) ansehen m"ochte.

Auf irgendeine n"ahere zoologische Deutung der Formen und deren Vergleichung mit irdischen hat sich der Verfasser nicht eingelassen.

Eine gro"se Anzahl dieser Bilder nun hat gewiss jeden Zoologen und Pal"aontologen aufs h"ochste "uberrascht. Ein f"ur Korallengebilde ge"ubtes Auge werden zun"achst die Bilder Tafel 1, Fig. 5 und 6, Tafel 8 und Tafel 15 an wohlbekannte Korallenstrukturen erinnern. W"are aber auch nur eine einzige dieser Formen sicher als eine organische nachgewiesen, so w"are der Bann gebrochen, und man d"urste dann zuversichtlicher auch an die organische Deutung der "ubrigen herantreten.

"Uber jene obengenannten, f"ur uns auffallendsten Formen nun, welche gr"o"stenteils von dem ber"uhmten kolossalen Meteorfall von Knyahinya in Ungarn (9. Juni 1866) herr"uhren, erlauben wir uns einige Worte.

Auf unseren Wunsch hat uns Dr. Hahn die Originalschlisse selbst, darunter ein "au"serst wertvolles Unikum zur n"aheren Untersuchung "uberlassen, und wir hatten volle Mu"se, diese merkw"urdigen St"ucke mit Zuhilfenahme unserer eigenen ziemlich reichhaltigen Korallensammlung zu studieren. Das Resultat dieser Untersuchung aber ist die volle "Uberzeugung, dass wir es, bei diesen Gebilden wenigstens, wirklich mit Resten von Korallen zu tun haben, die meist in die N"ahe der Favositinen geh"oren, einer Familie, die auf der Erde bis jetzt nur fossil und zwar in den pal"aolithischen [Pal"aozoikum], den "altesten Schichten der Erde gefunden worden.

Die irdischen Polypenst"ocke dieser Favositinen setzen sich aus parallel neben einander laufenden Polypenr"ohren zusammen. Von oben, wo die Kelche m"unden und die jeweilig lebenden Polypen sitzen, zeigt der Korallenstock des \emph{Favositen} ein mehr oder weniger regelm"a"siges Netzwerk, bestehend aus den W"anden der einzelnen Polypen. Au"serdem find besonders charakteristisch f"ur die \emph{Favositen} Querscheidew"ande in den Polypenr"ohren und ziemlich regelm"a"sig in Reihen stehende L"ochelchen in der Wandung, welche zur Herstellung der Gef"a"sverbindung der Polypenr"ohren unter einander dienen.

Solche Polyparien, d. h. ganz favositen"ahnliche R"ohrenb"undel, treten nun in einer gro"sen Anzahl der Meteoritenschliffe von Dr. Hahn auf, die von verschiedenen Meteorf"allen herr"uhren. Mit voller Klarheit aber sieht man an vielen derselben gerade auch jene Querscheidew"ande und L"ochelchen in Reihen und in bestimmten Distanzen voneinander und zwar so regelm"a"sig, dass hier an einen Zufall durchaus nicht zu denken ist, so wenig als irgend ein Mineraloge diese seinen Strukturverh"altnisse, Querscheidew"ande und L"ochelchen, die schon bei zweihundertfacher Vergr"o"serung deutlich find, die wir aber bis zu einer vierhundert achtzigfachen leicht verfolgen konnten, mineralogisch zu deuten versuchen wird. Wir haben es hier sicher mit organischen Gebilden, und zwar speziell mit favositen"ahnlichen Korallen zu tun.

Leider find die meisten Schliffe parallel der R"ohrenlage der Polyparien gegangen, was daher r"uhrt, da"s Hahn, um seine Schliffe zu erhalten, die Meteoritenmassen zerschlug, wo dann immer die Splitterung am leichtesten nach der L"ange des Polypenstockes erfolgte.

Nur ein einziger, wunderbar sch"oner Schliff, das obengenannte Unikum, gleichfalls von Knyahinya stammend, gew"ahrt als Querschliff durch den Korallenstock die volle Einsicht von oben in die Kelche des Polypariums und in die Aneinanderreihung der Kelche selbst. Dieses Pr"aparat allein schon ist gewiss f"ur jeden Korallenkenner entscheidend. Leider gibt die photographische Abbildung, die Hahn in seine Werke, Tafel 10, Fig. 3 und 4 gegeben, bei weitem nicht das klare Bild, wie es das Objekt selbst unter einem guten Mikroskop aufs deutlichste darlegt, indem eine gelbliche F"arbung des Pr"aparats bei der Photographie st"orte.

Dieses Objekt nun ist offenbar ein vollst"andiges kleines, rundliches Korallenst"ockchen, das mit einer breiten Basis auf einem anderen korallen"ahnlichen Gebilde auffitzt. Das ganze Netzwerk der Kelche tritt hier aufs klarste hervor. Die Kelche selbst find in der Mitte dunkel, mit einer schwarzen Masse ausgef"ullt, dann folgt eine wei"sliche F"ullmasse um jenen dunkeln Kern herum und schlie"slich aufs deutlichste die Wand jeder R"ohre, stets eine scharfe, schon bei geringer Vergr"o"serung sichtbare Linie, die sich bei st"arkerer Vergr"o"serung da und dort in zwei parallele Linien teilt, so dass jede Polypenr"ohre ihre eigene Wandung erh"alt. Dieses Netzwerk der die Polypenkelche scheidenden Linien zeigt weiter eine sehr verschiedene Gr"o"se und Form der Kelche. Die letzteren find n"amlich, ganz wie wir es bei einer Menge von Korallen und besonders auch bei dem devonischen \emph{Favosites polymorphus} beobachten, sehr unregelm"a"sig, bald mehr von gebogenen, bald von geraden Linien begrenzt, auch gr"o"ser oder kleiner, indem sich kleinere Kelche zwischen die gr"o"seren schieben oder durch Querteilung eines Kelches sich bilden, wie wir dies ja h"aufig bei den Korallen beobachten.

Alle diese Korallengebilde in den Meteoriten find verkieselt. Es find Magnesiumsilikate, daher man sie als Olivine deutete.

Noch aber ist eine "au"serst merkw"urdige Tatsache bez"uglich dieser au"serirdischen Korallengebilde zu konstatieren. Es ist dies die au"serordentliche Kleinheit. Es ist eine wahre Liliput-Tierwelt gegen"uber der irdischen. Das von uns soeben genannte Korallenst"ockchen, das wir bald an einem anderen Orte (zu Ehrenseines Entdeckers unter dem Namen \emph{Hahnia meteoritica}) ausf"uhrlicher beschreiben und abbilden werden, ist ein eben noch f"ur ein gutes Auge sichtbares, wei"ses T"upfelchen in dem Meteorschliff. Sein gr"o"ster Durchmesser misst nur 0,90 mm, die einzelnen Kelche durchschnittlich etwa nur 0,05 mm. Das find Verh"altnisse, wie wir sie von keinem irdischen Polypenstock kennen, wo Kelche von 1 mm Durchmesser schon sehr klein hei"sen. Doch werden wir uns auf noch ganz andere Dinge bei diesen au"serirdischen Tierorganismen gefasst machen m"ussen. Es k"onnen da leicht Formen vorkommen, die wir durchaus nicht in unser System der Erdzoologie einreihen k"onnen, ja, es erstaunt uns fast, dass wir in den oben genannten Gebilden Korallenformen vor uns haben, die einen so nahen Vergleich mit den irdischen zulassen. Es zeugt dies aufs sprechende von einer im gro"sen Ganzen immerhin au"serordentlich "ahnlichen organischen Entwickelung auf jenem oder jenen Planeten, von welchen jene Meteorite herstammen.

Noch k"onnte man wohl fragen, wie es kam, dass bei der gro"sen Anzahl von Meteoriten, die in den mineralogischen Sammlungen liegen, und bei der nicht unbedeutenden Anzahl von Forschern, die sich damit besch"aftigt, jene merkw"urdigen, organischen Gebilde noch nicht entdeckt worden find. Verschiedene Umst"ande m"ogen die Sache erkl"aren. Einmal find die Meteorite immer seltene und theuere St"ucke, die man nicht gerne opfert, daher im Ganzen immerhin wenig Schliffe gemacht wurden, so dass die Wahrscheinlichkeit, dass gerade ein g"unstiges Objekt bei diesen Schliffen zur Anschauung kam, nicht eben gro"s war. Hahn aber hat mit wirklich au"serordentlichen Opfern an Zeit und Geld nicht weniger als 600 Schliffe hergestellt. Sodann wurden dieselben meist nur mit der Loupe, selten mit st"arkeren Mikroskopen, und immer nur wenige Schliffe untersucht.

Dennoch haben einzelne Beobachter, besonders Direktor G"umbel in seiner Beschreibung der Meteorite von Eichst"adt und Sch"oneberg wahrscheinlich solche organischen Formen vor sich gehabt. Er beschreibt dort ausf"uhrlich und sehr gut s"aulenf"ormige Fasern, ja, er spricht sogar von unregelm"a"sig eckigen, kleinsten Feldchen, die bei Querschnitten durch jene Fasern entstehen. Hier hatte er wahrscheinlich solche kleine favositen"ahnliche Korallen vor sich, dachte aber dabei noch an keine Organismen. Doch sagt G"umbel wie vorahnend bei Besprechung des Meteorits von Kaba: "`Vielleicht gelingt es dennoch, die Anwesenheit organischer Wesen auf au"serirdischen K"orpern nachzuweisen."'

Wir glauben nach dem Obigen, dass dies in der Tat unserem unerm"udlichen Landsmann Dr. Hahn gelungen ist. W"are G"umbel durch einen gl"ucklichen Zufall auf ein St"uck wie jenes obengenannte Unikum getroffen, deren es wohl freilich noch viele in der zwei Center schweren Meteormasse von Knyahinya geben mag, so w"are er sicher der Entdecker dieser merkw"urdigen Tatsache geworden.

"Uber die Spongien und Krinoiden Hahns vielleicht ein anderes Mal!
\rule{\textwidth}{1.6pt}\vspace*{-\baselineskip}\vspace*{2pt} % Thick horizontal rule
\rule{\textwidth}{0.4pt} % Thin horizontal rule
\paragraph{}
Seit wir Obiges geschrieben, hat uns Herr Dr. Hahn s"amtliche, seinem Meteoritenwerke zu Grunde liegenden Schliffe und noch weitere neue, im ganzen "uber dreihundert, zur n"aheren, zoologischen Untersuchung und Bearbeitung "ubergeben. Es liegt hier eine gro"se F"ulle von Material vor, denn die Mehrzahl der Schliffe, die von Knyahinya z. B. find offenbar ihrem gr"o"sten Teile nach aus organischen Tr"ummern zusammengebacken. Gut erhaltene Formen find freilich ziemlich selten; es ist meist Detritus, wie man ihn z. B. im j"ungsten Meereskalk am mexikanischen Golf in Schliffen ganz "ahnlich beobachtet. Aber nachdem man sich einige "Ubung verschafft und viele Schliffe verglichen, lassen sich bald gewisse, stets wiederkehrende Formen recht wohl restituieren. Besonders entwickelt find die Schw"amme, von denen ich drei bestimmte Gattungen (Genera) bereits sicher festgestellt habe. Von einem sehr charakteristischen, bl"aulichen Schwamm, der h"aufig wiederkehrt, in jungen und alten Exemplaren, konnte ich nach einigen sehr g"unstigen Quer- und L"angsschlissen die innere Struktur so deutlich zeichnen wie von einem lebenden. Auch Pflanzenspuren scheinen vorzukommen; wenigstens erinnert ein sehr auffallendes, gew"olbt-schildf"ormiges, durch ein L"angscharnier zweigeteiltes Gebilde von 0,8 mm Durchmesser am ehesten an die Schildalgen, \emph{Cocconeis}. Ob die von Hahn in seinem Buche im allgemeinen als Krinoiden angesprochenen Formen wirklich dieser Klasse angeh"oren, scheint uns noch fraglich. Einige derselben find sicher Schw"amme. — Von einer h"oheren Tierform aber, von Weichtieren, Gliedertieren u. f. f. haben wir bis jetzt keine Spur gefunden; alle Formen repr"asentieren vielmehr offenbar eine sehr fr"uhe Formation des betreffenden Weltk"orpers. Die ganze vorliegende Tierformenwelt, die gewiss gegen f"unfzig verschiedenen Arten angeh"ort und die von verschiedenen Meteorf"allen, selbst solchen vom vorigen Jahrhundert herstammt, macht ferner ganz den Eindruck einer zusammengeh"origen Sch"opfung, die zweifelsohne von einem einzigen au"serirdischen Weltk"orper herr"uhrt. Die neueste Meteoritentheorie aber, die von dem ber"uhmten Schiaparelli herr"uhrt und die Meteorsteine mit den Kometen und deren Schweifen in Verbindung bringt, scheint nach Obigem nicht mehr haltbar. Alle jene Organismen haben im Wasser, und zwar in niemals ganz frierendem Wasser gelebt, das wir wohl auf den Kometen nicht suchen d"urfen. Auch dies zeigt die Tragweite der Hahn'schen Entdeckung, welcher eine zoologische Grundlage zu schaffen, uns zu gro"ser Freude gereichen wird.
\end{document}
