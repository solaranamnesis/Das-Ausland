\documentclass[a4paper, 12pt, oneside]{article}
\usepackage[utf8]{inputenc}
\usepackage[T1]{fontenc}
\usepackage[ngerman]{babel}
\usepackage{yfonts}
%\usepackage{fbb} %Derived from Cardo, provides a Bembo-like font family in otf and pfb format plus LaTeX font support files
\usepackage{booktabs}
\setlength{\emergencystretch}{15pt}
\usepackage{fancyhdr}
\usepackage{microtype}
\begin{document}
\swabfamily
\begin{titlepage} % Suppresses headers and footers on the title page
	\centering % Centre everything on the title page
	\scshape % Use small caps for all text on the title page

	%------------------------------------------------
	%	Title
	%------------------------------------------------
	
	\rule{\textwidth}{1.6pt}\vspace*{-\baselineskip}\vspace*{2pt} % Thick horizontal rule
	\rule{\textwidth}{0.4pt} % Thin horizontal rule
	
	\vspace{1\baselineskip} % Whitespace above the title
	
	{\LARGE Weiteres "uber die Tierreste in\\[0.25in] Meteoriten}
	
	\vspace{1\baselineskip} % Whitespace above the title

	\rule{\textwidth}{0.4pt}\vspace*{-\baselineskip}\vspace{3.2pt} % Thin horizontal rule
	\rule{\textwidth}{1.6pt} % Thick horizontal rule
	
	\vspace{1\baselineskip} % Whitespace after the title block
	
	%------------------------------------------------
	%	Subtitle
	%------------------------------------------------
	
	{Dr. David Friedrich Weinland} % Subtitle or further description
	
	\vspace*{1\baselineskip} % Whitespace under the subtitle
	
    {\small Das Ausland, Nr. 26, Artikel 1} % Subtitle or further description
    
	%------------------------------------------------
	%	Editor(s)
	%------------------------------------------------
	
	\vspace{1\baselineskip}

	{\small\scshape Stuttgart --- 27. Juni. 1881}
			
    \vspace*{\fill}

    Internet Archive Online Edition  % Publication year
	
	{\small Namensnennung Nicht-kommerziell Weitergabe unter gleichen Bedingungen 4.0 International} % Publisher
\end{titlepage}
\setlength{\parskip}{1mm plus1mm minus1mm}
\clearpage
\paragraph{}
Die kritischen Bemerkungen, die Herr A. Rzehak aus Br"unn in dieser Zeitschrift Nr. 20 "uber die Organismen der Meteorite ver"offentlichte, veranlassen mich noch zu einigen Worten in dieser Sache, da Herr Rzehak ausdr"ucklich sich auf meine Notizen "uber Korallen in Meteoriten in Nr. 16 dieses Organs bezieht.

Es ist sehr begreiflich, dass der Mineraloge, sobald es sich um einen "`Stein"' handelt, zun"achst sein Recht darauf wahrt und die Deutung der Entstehung desselben, sowie seiner Form, im Gro"sen und im Kleinen, als seine Aufgabe beansprucht. Niemand wird ihm das bestreiten, und wenn und solange er mit einer klaren, wissenschaftlich begreiflichen Erkl"arung zu Stande kommt, wird jeder gerne demselben Glauben schenken. Sobald aber die mineralogische Deutung eines "`Steins"' sehr schwierig wird, wie dies wohl betreffs der Chondren der Meteorite allseitig zugegeben wird, so liegt die Gefahr einer k"unstlichen, gezwungenen Deutung sehr nahe, w"ahrend vielleicht ein anderes naturwissenschaftliches Fach eine sehr nat"urliche und die allein richtige Erkl"arung geben k"onnte. Denken wir an die Geschichte der Petrefaktenkunde. Ist es doch noch gar nicht so lange her, dass man versuchte, die versteinerten Tierreste, eben weil es Steine waren, auf alle m"ogliche Weise, sogar als "`Naturspiele"', nur nicht auf die nat"urlichste und allein richtige Art zu erkl"aren, --- bis die Zoologie die Sache in die Hand nahm und die Pal"aontologie schuf, bekanntlich nicht ohne anf"angliche heftige Widerspr"uche. Man denke nur an die "`rires unanimes"' der von Herrn Rzehak angerufenen franz"osischen Akademie, als Cuvier im Anfang dieses Jahrhunderts dort die fossilen Elefanten begr"undete. Ganz ebenso wird es mit den Chondritmeteoriten und ihren Einschl"ussen ergehen. Aber es werden keine zehn Jahre vor"uber sein und wir werden eine kleine, allseitig anerkannte Fauna der Meteorite besitzen. Das scheint vielleicht heute noch ein gewagtes Wort, aber meine Fachgenoffen, die mich seit f"unfundzwanzig Jahren kennen, werden wohl glauben, dass ich diese meine Überzeugung nicht leichthin ausspreche. --- Doch zur Sache.

Das Dr. Hahn'sche Meteoritenwerk, das sich auf hunderte von Meteoritenschliffen, die von achtzehn verschiedenen Meteorf"allen herr"uhren, gr"undet und das eine erste deutsche Autorit"at, Professor R. [?], "`mag die Deutung sein, welche sie will, ein ausgezeichnetes Prachtwerk von gr"o"stem, wissenschaftlichem Nutzen"' nennt, versucht Herr Rzehak aus Br"unn kurz abzutun zun"achst mit Berufung auf einen franz"osischen Mineralogen, der fr"uher gleichfalls "uber Meteoriten schrieb und sich nat"urlich von Dr. Hahn, dem "`savant allemand"', der, wenn auch ein allseitig anerkannter, t"uchtiger Mineraloge und ausgezeichneter Mikroskopiker, doch nicht eigentlich Fachmann von Profession ist, --- die Mangelhaftigkeit seiner Beobachtungen und besonders seiner Abbildungen nicht gerne nachweisen lassen konnte. Sodann bereust sich Herr Rzehak des Weiteren auf seine eigenen Beobachtungen an einigen wenigen Meteoritenschliffen von dem Fall von Tieschitz in M"ahren, in denen er alles Material gefunden zu haben glaubt, um das ganze Werk von Dr. Hahn \emph{ad absurdum} zu f"uhren. ---

Gewiss ist jeder Fachmann zuerst mit gro"sen Zweifeln an dieses Werk herangetreten. Die Sache kam zu unerwartet. Manche der von Hahn abgebildeten Formen mussten zwar jedem Kenner der mikroskopischen, tierischen Struktur sofort als typisch organisch auffallen, aber die Herkunft der Gebilde ebenso sehr zur Vorsicht mahnen. So hat denn auch unsren Wissens kein deutscher Forscher bis jetzt ein unbedingtes, positives oder negatives Urteil, zumal "offentlich, lediglich nach Ansicht des Werkes, ohne Einsichtnahme der Objekte selbst, auszusprechen gewagt. ---

Die oben ber"uhrten Notizen im \emph{Ausland} "uber Korallen in Meteoriten wurden von mir geschrieben, als ich erst einige wenige, speziell von mir gew"unschte Schliffe studiert hatte. Nachdem ich nun seitdem die reiche Hahn'sche Meteoritensammlung seit Monaten zu meiner Verf"ugung gehabt, hatte ich nicht nur Gelegenheit, die von ihm abgebildeten, sondern auch eine gr"o"sere Anzahl neuer, speziell f"ur den Zoologen noch weit beweisender er St"ucke zu finden. Die Tatsache, dass wir es bei den Chondritmeteoriten, bei den einen weniger, bei den anderen mehr, mit einer Menge organischer Einschl"usse und zwar ganz verschiedenen Familien, ja Klassen angeh"origer, tierischer Reste zu tun haben, steht "uber jeden Zweifel erhaben. Eine kurze Zusammenstellung der Resultate meiner bisherigen Studien, in der ich eine Anzahl von Gattungen und Arten kurz charakterisieren, auch einige Abbildungen beigeben werde, soll im Laufe des Sommers in der \emph{Leopoldina} erscheinen und liegt bereits bei dieser Akademie. Eine gr"o"sere Arbeit f"ur die \emph{Acta} derselben Akademie, mit ausf"uhrlichen Strukturbeschreibungen und Zeichnungen ist in Vorbereitung. Auf diese beiden k"onnte ich f"uglich verweisen, allein in unserer schnelllebigen Zeit l"asst man sich nicht gerne auf Zuk"unftiges vertr"osten, darum erlaube ich mir, hier sogleich noch einiges anzuf"uhren, schicke aber ausdr"ucklich voraus, dass mein Standpunkt in der Sache ein vollkommen unparteiischer ist, dass ich mich in meiner Deutung der Formen und der Resultate in keiner Weise durch die fr"uheren Deutungen Dr. Hahns in seinem Meteoritenwerke oder seine Schlussfolgerungen gebunden f"uhle, wor"uber ich mich auch mit meinem Freunde Dr. Hahn vollkommen verst"andigt habe, als ich in die zoologische Bearbeitung seiner Entdeckung eintrat. F"ur mich handelte es sich von Anfang an lediglich um die Fragen: Sind die betreffenden Gebilde organische Formen, welcher Art sind sie, verglichen mit den irdischen und auf welche unmittelbare Folgerungen l"asst ihre Anwesenheit in den Meteoriten f"ur die Herkunft dieser selbst schlie"sen?

Nun einige Punkte:
\begin{enumerate}
    \item Die verschiedenen Chondritmeteoriten sind sehr ungleich reich an organischen Gebilden, manche bestehen zu zwei Dritteilen oder mehr daraus. In der Regel sind es kleinere oder gr"o"sere Bruchst"ucke und meist erst, wenn man eine gr"o"sere Anzahl von Schliffen durchmustert hat, findet man die Zusammengeh"origkeit der verschiedenen Gebilde heraus, ganz wie es bekanntlich auch bei selteneren irdischen Petrefakten der Fall ist. "`Prachtexemplare von Organismen"', wie Herr Rzehak sie gleich in seinen ersten und einzigen M"ahrischen Schliffen gefunden, find leider ziemlich selten. Wir haben deren in unseren sechshundert Schliffen nur vielleicht ein Dutzend. Unter solchen verstehe ich n"amlich vor allem jene Formen, wo ein gro"ser Teil der "au"seren Konturen des tierischen Organismus zugleich mit der inneren Struktur zur Anschauung kommt. So habe ich z. B. eine Schwammform gefunden, und zwar gerade diese in einer Anzahl von St"ucken, wo nicht nur die "au"sere Form, ein flacher, nach allen Seiten hin abgerundeter Lappen, sondern auch durch zuf"allige, gl"uckliche Schliffe, die por"ose Rindenschicht des Schwammes von oben und im Durchschnitt, sodann das Maschenskelet des Gastrovascularsystems, das den Schwamm ausf"ullt, aufs vollkommenste erhalten ist, so gut wie in irgend einem irdischen Petrefakt. Diese Formen gedenke ich --- mit Verlaub des Herrn Rzehak, dem meinem Gattungsnamen \emph{Hahnia} nicht besonders zu gefallen scheint, --- \emph{Pectiscus} zu nennen. Anderen Schwammformen, gleichfalls in gro"ser Anzahl, aber mit einer anderen, feineren Rindenschicht, auch einem anderen, sehr merkw"urdigen, sternf"ormige Maschen bildenden, Gastrovascularsystem werde ich den Namen \emph{Urania} zu belassen vorschlagen, den Hahn urspr"unglich f"ur diese Form geschaffen, freilich, als er diese Gebilde alle noch f"ur Pflanzen ansah, woran Herr Rzehak so gro"sen Ansto"s nimmt, vielleicht aber meinen Freund Dr. Hahn eher entschuldigt, wenn er sich erinnert, dass die Schw"amme noch im Anfang diese Jahrhunderts von vielen t"uchtigen Forschern f"ur Pflanzen erkl"art wurden. Ich erlaube mir, hier "uberdies anzuf"ugen, dass f"ur Dr. Hahn, wie er ausdr"ucklich in seinem Buche erkl"art, die zoologische Einreihung seiner Formen nicht die Hauptsache war und nicht sein konnte, weil er eben in der Zoologie nicht Fachmann ist. Ihm war es nur darum zu tun, nachzuweisen, dass organische Gebilde in den Meteoriten vorhanden find und dies ist und bleibt sein gro"ses und bedeutungsvolles Verdienst, mag es mit seinen zoologischen Deutungen, besonders von Krinoiden u. f. w., denen ich durchaus nicht "uberall folgen kann, gehen, wie es will.
    \item Es handelt sich durchaus nicht allein, wie Herr Rzehak anzunehmen scheint, um die Erkl"arung jener von Direktor G"umbel schon so gut beschriebenen, faserigen oder s"aulchenf"ormigen Gebilde, die Herr Rzehak in seinen M"ahrischen Meteoriten auch gefunden, ja sogar in einem fraglichen Feldspat beobachtet haben will, und deren Querscheidew"ande er f"ur "`Querkl"ufte"' erkl"art (unsere Instrumente zeigen aber nicht Kl"ufte, sondern deutliche, k"orperliche Scheidenw"ande), vielmehr au"ser ihnen noch um eine gro"se Anzahl anderer, ganz verschiedener Gebilde, die mit jenen faserigen (d. h. in Wirklichkeit in parallelen R"ohren angeordneten) nicht das mindeste zu tun haben, z. B. neben den vorhin erw"ahnten Schwammformen \emph{Pectiscus} und \emph{Urania} noch um ein hahnenkammf"ormiges Gebilde, das wohl zu den Foraminiferen geh"oren wird und uns aufs lebhafteste an die \emph{Carpenteria rhaphidodendron} von M"obius erinnert, weiter um facettierte K"ugelchen, die aus Kieselb"alkchen regelm"a"sig "uber einander geschichtet find, welche B"alkchen selbst hohl und mit L"ochelchen versehen find und die ich nur mit jenen zierlichen Radiolarienskeleten vergleichen kann, wie sie H"ackel in seinem sch"onen Werke abgebildet. (Dr. Hahn hatte sie vorl"aufig zu den Krinoiden gestellt. Über die "ubrigen, sogenannten Krinoiden Hahns, die unserem Herrn Rzehak ganz besonders zu schaffen machen, werde ich am betreffenden Orte ausf"uhrlicher referieren). Weiter handelt es sich um andere, wahrscheinlich auch zu den Radiolarien geh"origen Formen, deren Kieselb"alkchen in der Peripherie in ein Maschennetz "ubergehen, sodann wieder um andere schildf"ormige, deren Beschreibung ohne Abbildungen doch keine klaren Begriffe geben w"urde u. f. f.
    \item Der erste Eindruck, den man bei der Messung dieser Meteoritformen erh"alt, ist der einer au"serordentlichen Kleinheit, wie schon Hahn hervorhob und wie ich dies auch in jenen, meinen ersten Notizen im \emph{Ausland} bemerkt habe. Seit aber eine gr"o"sere Anzahl der Formen als Foraminiferen und Radiolarien erkannt find, deren Gr"o"se ganz gut zu der der irdischen Formen stimmt, bleiben eigentlich nur noch die Korallen der Meteorite als ungew"ohnlich kleine Gebilde "ubrig. Doch auch bei diesen ist das Verh"altnis kein so ganz au"serordentliches. Auch von irdischen Korallen kennt man Kelche von 1 ja sogar 1/2 mm Durchmesser, w"ahrend die der Meteorite bis zu 0,1 mm messen. Ebenso gibt es auch von irdischen Schw"ammen bekanntlich mikroskopisch kleine Arten. Wenn wir au"serdem bedenken, dass wir bei diesen Meteoriten immer nur mit D"unnschliffen arbeiten, so ist verst"andlich, dass gr"o"sere Formen nicht leicht --- auch nur als Bruchst"ucke --- gut zahllosen Struktursetzen, die wir in den Schliffen beobachten, gr"o"seren Formen angeh"oren d"urften.
    \item Ein gro"ses Missverst"andnis w"are die Annahme, die mir k"urzlich in einem Briefe eines bedeutenden Schriftstellers entgegentrat und die vielleicht auch Anderen, welche die Zusammensetzung der Chondritmeteoriten nicht kennen, sich nahe gelegt haben m"ochte, dass n"amlich jene organischen Formen Reste niederer Tiere sein k"onnten, die au"sen auf der Oberfl"ache jener Meteoriten w"ahrend ihres Laufes durch den Weltraum entstanden w"aren. Dies ist nat"urlich durchaus nicht der Fall. Jene Gebilde sind vielmehr Einschl"usse in den Meteoriten. Es sind Petrefakten, nichts anderes und die Chondritmeteoriten selbst erscheinen uns bis jetzt in der Tat lediglich als Petrefaktenf"uhrende Felstr"ummer eines fremden Weltk"orpers, aber wahrlich als solche interessant genug. ---
\end{enumerate}
\paragraph{}
Herrn Rzehak aus Br"unn aber ersuchen wir freundlichst, ehe er sich weiter in der Sache vernehmen l"asst, entweder um g"utige Einsichtnahme unserer Schliffe selbst oder aber --- um geduldiges Weiterschleifen und dann um Weitermikroskopieren, wie Hahn und Ich es Monate lang getan. Wer wei"s, er wird dann vielleicht auf die eine oder andere Art ein Verfechter der Hahn'schen Entdeckung, wie es erst vor kurzem ein wohlbekannter s"uddeutscher Mineraloge und Pal"aontologe an meinem Mikroskop geworden.
\rule{\textwidth}{1.6pt}\vspace*{-\baselineskip}\vspace*{2pt} % Thick horizontal rule
\rule{\textwidth}{0.4pt} % Thin horizontal rule
\paragraph{}
Bei der ungeheuren Wichtigkeit, welche die in Obigem er"orterte Frage f"ur die moderne Naturwissenschaft besitzt und dem regen Interesse, welches die Diskussion dar"uber in den beteiligten Kreisen erregt, glaubt die Redaktion den vorstehenden Ausf"uhrungen Dr. Weinlands die folgenden Bemerkungen Dr. O. Hahns selbst unmittelbar anreihen zu sollen. Dr. O. Hahn schreibt:

In der Nr. 20 des \emph{Ausland} tritt Herr Anton Rzehak in Br"unn gegen die "`Organismen der Meteorite"' auf.

Seine Beweise sind in der Hauptsache folgende:
\begin{enumerate}
    \item Die Akademie in Paris hat die Sache nicht anerkannt.
    \item Die Merkmale eines organischen Wesens find von Hahn nicht richtig angegeben, weil --- zwei der f"unf von ihm angegebenen Merkmale f"ur sich noch kein Beweis eines Organismus find.
    \item Es gibt --- und hier beruft sich Herr Rzehak auf ein mit einem Fragezeichen versehenes Mineral (Feldspat?) welches einen ziemlich deutlichen, s"auligen Bau, aber freilich in diesen S"aulen keine radiale Anordnung habe --- auch R"ohrenbildungen im Mineralreich, also m"ussen, schlie"st er, die R"ohren in den Chondriten nicht notwendig organischen Ursprungs sein.
    \item Enstatit und Turmalin haben Querkl"ufte, welche leicht mit Querscheidew"anden von Organismen verwechselt worden sein k"onnen.
    \item In dem Mineral "`Feldspat (?)"' sah Herr Rzehak mehrere in einer L"angsreihe geordnete Einschl"usse: folglich sind, schlie"st er, "`offenbar"' solche Einschl"usse in Chondritmineralien irrt"umlich f"ur "`Perforationen"' gehalten worden.
    \item Auch Hagelk"orner kommen vor, welche Ähnlichkeit mit der Struktur der Chondrite haben. (G"umbel.)
\end{enumerate}
\paragraph{}
Was weiter vorgebracht ist, ist blo"s Kritik von Schlussfolgerungen, welche ich deshalb bei Seite lasse, weil, wenn meine Tatsachen richtig sind, diese Kritik von selbst f"allt.

Was nun die Autorit"at der Pariser Akademie betrifft, so bemerke ich nur, dass es dieselbe Akademie ist, welche neun Jahre lang nach dem Erscheinen von Chladnis Buche "uber den kosmischen Ursprung der Meteorite noch die Behauptung des Fallens von Meteorsteinen f"ur einen Wahnsinn erkl"arte, dann aber doch, freilich erst nach neun Jahren, durch einen --- Postknecht sich von der Unrichtigkeit ihrer bisherigen Meinung "uberzeugen lie"s. Ihr Trost waren damals folgende S"atze: "`Der Dumme glaubt,"' "`der Halbgebildete entscheidet,"' "`der Gebildete pr"uft,"' gewiss ein leichter Trost f"ur solche Irrt"umer (Quenstedt, \emph{Klar und Wahr} S. 287).

Wenn sich Herr Rzehak auf das Urteil der \emph{Comptes rendus} beruft, so muss ich beif"ugen, dass das Mitglied der Pariser Akademie, Herr Daubrée, (nicht Dumas) welcher mein Werk entgegennahm, mir erwiderte: er habe durch Schmelzung "ahnliche Formen, wie ich in den Chondriten fand, erhalten; auf meine Bitte um Mitteilung eines solchen Schmelzprodukts aber erhielt ich weder Antwort noch viel weniger ein solches Produkt: ein Verfahren, welches nicht f"ur die Richtigkeit einer Behauptung spricht.

Herr Daubrée hat in seinem Buche \emph{Experimentalgeologie} S. 386 den Meteorstein von Knyahynia abgebildet, freilich sehr wenig genau. Dass die Einschl"usse Strukturen haben, hat er "ubersehen, aus dem einfachen Grunde, weil alle seine Untersuchungen mit Pulvern und Schmelzen der Steine anfingen.

Auch die Akademie von 1800 hatte noch Hunderte von "`Arguments physiques et moraux"' gegen den kosmischen Ursprung der Meteorite, eine Ansicht, welche, wenn sie heute wiederholt w"urde, auch keinen anderen Erfolg, als den des Gel"achters h"atte.

Herr Rzehak aber hat "`Arguments physiques et moraux"' gegen meine Arbeit, welche ich nun des N"aheren er"ortern will.

Er bestreitet vor allem meine Definition des organischen Wesens, indem er zwei Merkmale meines Begriffs, n"amlich: "`geschlossene Form"' und "`wiederkehrende Form"' f"ur sich allein nicht hinreichend l"asst, um den Beweis des Vorhandenseins eines Organismus zu f"uhren. Da ich aber f"unf zusammengeh"orige Merkmale f"ur den Beweis eines organischen Wesens fordere, so habe ich selbst diese beiden Merkmale allein f"ur unzureichend zum Beweis erkl"art: als Argument gegen mich ist dieser Einwurf also kein ehrlicher.

Mit den Einw"urfen 3, 4 und 5 will der Verfasser der Kritik die Struktur der Chondrite aus der der Minerale erkl"aren, wobei er sich gelegentlich auf G"umbel bereust.

Auf die einzelnen Beweise, sowohl den negativen, dass man es nicht mit Mineralbildungen, als auf den positiven, dass man es mit wirklichen Organismen zu tun habe, findet es Herr Rzehak einzugehen nicht n"otig: auch meine 32 photographischen Tafeln existieren f"ur ihn nicht. Dass diese denn doch f"ur sich schon ihre Bedeutung haben, daf"ur berufe ich mich auf das Urteil der ersten Autorit"at auf dem Gebiete der Mineralstruktur, welches solgenderma"sen lautet: "`Mag auch die Deutung sein, welche sie will, jedenfalls muss Ihr Buch als ein ausgezeichnetes Prachtwerk "uber die Meteoritenstruktur gelten, dessen Tafeln f"ur Jedermann von dem gr"o"sten wissenschaftlichen Nutzen find."'

Was ist aber das Beweismaterial des Herrn Rzehak? Ein (!) Mineral, das er nicht einmal zu bestimmen vermag --- ein Beweis, dass entweder das Mineral ganz unsicher und daher kein Beweismaterial oder Herr Rzehak kein Mineraloge ist. F"ur letzteres spricht, dass ihm sein (?) Mineral als Unikum erscheint, w"ahrend allerdings bei Feldspat eine Art H"ohlung (in Folge der Zersetzung) eine ganz gew"ohnliche Erscheinung ist. Es h"atte hier dieses dunkeln Kristalls gar nicht bedurft, Herr Rzehak h"atte sich geradezu und kurzweg auf diese Tatsache berufen k"onnen, nur freilich w"urde ich ihn dann auch auf den Unterschied zwischen Mineral und Organismus hingewiesen haben. Von diesem (?) Mineral giebt Herr Rzehak keine Abbildung, so dass sich die Leser ein Bild davon machen, dass sie selbst urteilen k"onnen.

Von dem Allem ist keine Rede. Man schlie"st \emph{a priori}, wo Tatsachen vorliegen, man bereust sich auf Minerale, welche niemand sehen und vergleichen kann, und macht sich luftig "uber Dinge, welche man offenbar nicht kennt.

Wenn ich Herrn Rzehak ein Urteil in der Sache zugestehen wollte, m"usste er mich versichern k"onnen, dass er mein Material kennt oder ebenso viel Material wenigstens gesehen hat, als ich. Doch zur Sache!

Mein Beweis ist zun"achst ein negativer, d. h. der Beweis, dass eine Mineralstruktur nicht vorliegt: und dann ein positiver, dass die Formen der Meteorite nach Analogie anerkannter Organismen dies auch find.

Der erste Beweis, der negativ mineralogische, ist (wie ich sagte) in der Kritik ganz "ubergangen. Ich h"atte vor allem eine Widerlegung dieses Teils erwartet, weil dieser jedermann zug"anglich ist. Ich kann mich hier auf mein Buch der Meteorite (Chondrite) S. 20 f. berufen, und bitte dort nachzulesen.

Nun m"ochte ich nur die eine Frage von dem Herrn Rzehak, wenn er ein Mineraloge ist, beantwortet h"oren: Wie ist es m"oglich, dass ein oder zwei Minerale, aus welchen die Chondrite nach der allgemeinen Annahme bestehen, Minerale aus demselben Steine (von etwa f"unf Zentner), also unter den gleichen Verh"altnissen entstanden und gebildet, alle die Hundert Strukturformen zeigen, welche ich in meinem Werke abgebildet habe? Und diese Strukturen k"onnten jetzt um 25 noch vermehrt werden.

Eine Antwort auf diese schon in meinem Buche aufgestellte Frage hat Herr Rzehak nicht gegeben: er begn"ugt sich, G"umbel anzuf"uhren, der in Eis (Hagelk"ornern) "ahnliche Strukturen wie in den Chondriten gefunden zu haben glaubte. --- Es w"are in der Tat schon ein gro"ses Wunder, wenn Eis- und Enstatit-Kristallite gleich Aufsehen. Dass es scheinbar stenglige Strukturen in Eis, dass es viele stenglige Mineralaggregate gibt, ist sicher, nur besteht eben der Unterschied, dass in den Chondriten nicht blo"s Bruch- (optische) Linien find, sondern wirklich substanzielle, von einem zweiten Mineral gebildete W"ande; dass diese "`Stengel"' nicht wirr durcheinander liegen, wie das (?) Mineral des Herrn Rzehak, sondern ganz regelm"a"sig aneinander geordnet find, und zwar exzentrisch und nicht konzentrisch, und dass ferner ein Teil nicht eine Kugel, sondern einen platten F"acher von R"ohren bildet. Der Kernpunkt meiner Beweisf"uhrung, der Schl"ussel meiner Stellung, find aber die regelm"a"sigen Strukturen im Gro"sen und kleinen, die Regelm"a"sigkeit derselben, welche die Annahme eines Naturspiels absolut ausschlie"st.

Ich habe deshalb eine Anzahl solcher mit h"oheren Vergr"o"serungen, wie Taf. 9, 15 gegeben; auch habe ich in den Texten noch erg"anzt, was ich wenigstens durch die Photographien bei so hohen Vergr"o"serungen nicht mehr zeigen konnte.

Was der Verfasser gegen diese photographischen Bilder aus seiner Beobachtung gegen die Struktur der Chondrite anf"uhrt und beschreibt, ist, wie oben schon ausgef"uhrt, ein Mineral mit einem Fragezeichen; er h"alt es f"ur Feldspat. In einem Mineral (welches? wei"s der Herr Verfasser ja selbst nicht), hat er "`s"auligen Bau"' beobachtet. Aber wohl gemerkt, er fand erstens nicht runde S"aulen, wie meine Formen find, sondern geradlinige Umrisse. Ebenso gut und weit einfacher h"atte er sich auf die Basalts"aulen zum Gegenbeweis berufen k"onnen.

Die Tatsache, dass meine Strukturen runde R"ohren find, wird entweder von Herrn Rzehak "ubersehen oder verschwiegen, und beides w"are notwendig, letzteres doppelt notwendig zu sagen gewesen, weil mein Buch, wie der Herr Verfasser selbst sagt, nur in wenigen H"anden ist, seine Kritik aber in viele H"ande kommt. Also R"ohren!

Zur Widerlegung meiner Ansicht h"atte geh"ort, dass er in seinem (?) Mineral R"ohrenstruktur nachgewiesen h"atte. Dass es Kristall-Aggregate, Kristalle mit geradlinigen Umrissen gibt, dazu bedarf es keines Minerals mit einem Fragezeichen: das wei"s jedermann, selbst der Laie. Dass es aber Minerale (und nicht Aggregate) gibt, welche ganz aus runden R"ohren bestehen, das habe ich weder geh"ort, noch gelesen, noch gesehen.

Ein blo"ses Mineral hat "uberhaupt keine Struktur, es kann sich blo"s in Folge von mechanischer Gewalt oder chemischer Zersetzung eine Art Struktur nachbilden, aus welcher r"uckw"arts auf das urspr"ungliche Mineral geschlossen wird. Also die Beobachtung an dem Feldspat in Frage trifft hier "uberhaupt nicht zu.

Was die Kristalls"aulen von den runden R"ohren der Chondrite unterscheidet, habe ich in meinem Buche angef"uhrt: es find Substanzen, welche die R"ohrenw"ande bilden und dabei eine F"ullmasse, also zwei Minerale, welche die R"ohren ausmachen, w"ahrend die Kristalls"aulen nur aus Einem Mineral bestehen und nur durch Spr"unge (also optische Linien) sichtbar werden. Weiteres find diese "`S"aulen"', wie Herr Rzehak zugibt, nicht radial angeordnet, wie die der Chondrite, sondern wirr, und es bedurfte nur eines Blickes in das Polarisations-Mikroskop, um den Unterschied beider Bildungen sofort im vollen Lichte zu zeigen. Überdies kommen, wie ich oben anf"uhrte, f"acherartig gelagerte R"ohren vor: und zwar blo"s aus einer Reihe R"ohren gebildet, die streng (ex-)zentrisch aneinandergelagert find. Freilich ist es eine leichte Sache, mit solchen Objekten und solchen Tatsachen, wie Herr Rzehak, zu "`beweisen"', mit Objekten, wobei man sicher ist, dass der Leser weder das eine Objekt des Angegriffenen noch das des Angreifers zu Gesicht bekommt; solche Dinge liest sogar der Fachmann in gutem Glauben vor, und "ubersieht leicht die Unterschiede, weil er auch nicht einmal das Buch des Gegners vor sich hat. Eine solche Beweisf"uhrung ist entweder unverzeihlich oberfl"achlich --- oder --- (wenn wissentlich) eine unehrliche.

Nun wird drittens, um auch die feineren Strukturen, die "`favositoiden"' Kan"ale mineralogisch zu erkl"aren, oder richtiger, um mir eine T"auschung auch in dieser Richtung nachzuweisen, sich auf Glas-Einschl"usse in dem (?) Mineral berufen, welche den Eindruck eines Querkanals, und wo sie sich aneinanderreihen, den eines Hohlraums machen k"onnen.

Nun habe ich aber in meinem Buche gesagt und gezeigt, dass die Kan"ale der Meteoriten-(\emph{Favositen}) in v"ollig gleichen Abschnitten stehen, nicht Glas-Einschl"usse sind, und will hier nachtragen, dass dieselben auch im Querdurchschnitt und zwar hier nicht als Punkte, sondern als deutliche Querkan"ale vorhanden sind, dass sie also nicht eingesprengte Minerale (Punkte), sondern wirklich ganz unzweifelhafte R"ohrchen (Sprossenkan"ale der \emph{Favositen}) sind. Hienach kann von runden Glas-Einschl"ussen als Gegenbeweistatsache durchaus nicht mehr die Rede sein. Noch kein Forscher, welcher meine Objekte gesehen, hat dem, was ich f"ur Sprossenkan"ale (Perforationen) erkl"arte, den Einwurf gemacht, dass es blo"se Einschl"usse seien.

Hier muss ich noch weiter gehen und auf den gr"o"sten Fehler aller Kritik "uber fremde Beobachtungen hinweisen: er besteht darin, dass man "uberhaupt Beobachtungen Dritter kritisiert, ehe man das Beobachtungsobjekt des Gegners gesehen hat.

Und gerade f"ur den vorliegenden Fall musste ich wenigstens darauf bestehen, dass die Kontrolle an Schliffen von Knyahynia ausgef"uhrt wird.

Ich kann den Herrn Verfasser versichern, dass ich schon Hunderte und Taufende von Glas-Einschl"ussen gesehen habe, aber kein Gestein hat nur entfernt das gezeigt, was ich in den Chondriten beobachtet habe. Hier find es bei einer Vergr"o"serung von 1000 nicht etwa Magnetitk"orner, wie sie auch in den Meteoritgesteinen h"aufig vorkommen, nicht beliebig geformte Glas-Einschl"usse, sondern kreisrunde, zuweilen elliptische geformte Fl"achen mit einer Wand, wenigstens mit einer dunkler gef"arbten Masse zwischen dem Kreis und dessen Umgebung; ferner liegt dieser Kreis h"aufig sogar in einer Vertiefung (was man schon auf Taf. 15 angedeutet sieht): die "`Perforationen"' finden sich nur in R"ohren und schlie"slich ist die Wand der Zollw"ande noch seitlich durchbohrt von den Kan"alen, welche symmetrisch und in gleichen Abst"anden zu denjenigen stehen, welche man im Querdurchschnitt als Punkte sieht. Diese Seitendurchschnitte find in der Form Taf. 8 bei 300-facher Vergr"o"serung schon ganz deutlich. Das ist denn doch etwas anderes als eine Einstreuung oder ein Einschluss.

Nun kommt der Herr Verfasser zum Schluss viertens an die Erkl"arung der Querscheidew"ande. Auch hier ist seine Kritik nicht minder unzutreffend. Er beruft sich darauf, dass die Enstatitkristalle Querbr"uche zeigen.

Das ist mir wohlbekannt. Ich habe aber in meinem Buche diesen Einwand sowohl, was die Erkl"arung der R"ohren, Lamellen aus Bl"atterbr"uchen, als was die Erkl"arung der Querscheidew"ande aus Querspr"ungen betrifft, er"ortert, und darauf hingewiesen, dass beide, Bl"atterbruch und Querspr"unge, blo"s optische Erscheinungen seien, w"ahrend die Zellenscheidew"ande der Organismen und gerade die Querscheidew"ande in meinen Formen aus besonderen Substanzen aufgebaut find. Ich habe deshalb, um das Bild von blo"sen Br"uchen und Scheidew"anden zu zeigen, einen terrestrischen Enstatit (Texas), der blo"s Mineral ist, abgebildet, wo diese Br"uche als schwarze Linien hervortreten.

Nun stimmt aber der Enstatit aus dem Meteorstein von Bishopsville, welcher ebenfalls reines Enstatitmineral ist, in seinem Bilde mit dem von Texas Taf. 1, Fig. 2 (also ein meteoritischer Enstatit mit einem terrestrischen Enstatit) so vollkommen "uberein, dass sich beide Bilder nicht unterscheiden lassen. Hat der meteoritische Enstatit, wo er blo"s als Mineral austritt, dieselbe Struktur, wie der terrestrische --- so folgt daraus auch, dass, wenn in die meteoritischen Minerale ganz andere Strukturen auftreten, letztere eine besondere (nicht in dem Mineral gelegene) Ursache haben m"ussen.

Hier muss ich noch eine freilich l"angst bekannte Tatsache anf"uhren.

Wo ein Organismus "`versteinert"' wird, tritt ein Mineral an die Stelle der organischen Stoffe. Es mag etwas von der urspr"unglichen Substanz zur"uckbleiben, wie z. B. die Kieselger"uste der Schw"amme. Doch das kommt hier nicht in Betracht. Gew"ohnlich wird die ganze Substanz umgeformt, jedenfalls Hohlr"aume wieder ausgef"ullt. Das umwandelnde Mineral ist Mineral und bleibt es, hat als solches seine Eigenschaften, sie k"onnen sich in der Art an die Stelle des urspr"unglichen Organismus dr"angen, dass nur dessen "au"serste Umrisse erhalten bleiben, w"ahrend die ganze Form mit dem Mineral ausgef"ullt ist. Dort liegt also nun z. B. Kalkspat mit seinen drei Bl"atterbr"uchen, an der Stelle des urspr"unglichen Cidaritenstachels in der Form desselben, wie ihn Quenstedt, \emph{Epochen der Natur}, S. 558 zeigt. Dieser Cidaritenstachel ist seiner Substanz nach reinem Kalkspat und zeigt an seiner Oberfl"ache nur eine etwas dunkler gef"arbte Substanz, jedoch ganz vollkommen erhaltene Umrisse, so dass niemand blo"s Kalkspat mit Bl"atterbr"uchen darin vermutete. Ganz so ist zum Teil die Art der Versteinerung in den Chondritorganismen. Au"sen Enstatit, innen Olivin. Aber auch da, wo die Strukturen erhalten find, find sie nur mit dem Mineral ausgef"ullt und dieses hat alle seine gew"ohnlichen physikalischen Eigenschaften. Notwendig also treten die Mineraleigenschaften (Mineral-Strukturen) zu den Überbleibseln der organischen Masse und Struktur und deshalb wird es immer so sein: Wenn man blo"s der ersteren erw"ahnt, --- der mineralogischen Erscheinungen --- so wird der Widersacher, der behauptet, es ist blo"s Mineral --- f"ur den Augenblick wenigstens recht behalten. Sobald man aber eine Erkl"arung der wirklich organischen Strukturen von ihm verlangt, wird ihn seine Kunst verlassen. Freilich behilft er sich dann gern mit allgemeinen Redensarten, z. B. "`erinnert doch"', "`ist analog, wenn auch nicht gleich"', "`zeigt Verwandtschaft"' und dergleichen. Solche Redensarten haben, wo wirklich eine Analogie da ist, eine Berechtigung. Aber auch die Analogie hat ihre wissenschaftliche Grenze, sonst k"onnte schlie"slich eine Taube auch an einen Dachziegel "`erinnern"'. Hier muss dann eben die genaueste Beobachtung der Merkmale und Vergleichung im Einzelnen eintreten. Hinsichtlich der Meteoritformen aber k"onnte als Analogie nur der terrestrische Enstatit und Olivin zugelassen werden, keineswegs aber Eis, ein beliebiger Feldspat zc.: streng genommen aber muss, sobald Enstatit- und Olivinstruktur in den Meteoriten wie in den terrestrischen Vorkommnissen vorhanden ist, eine Berufung auf andere Minerale wegfallen: hier beweist die Analogie gerade selbst sofort, dass man es nicht mit Mineralstruktur zu tun hat. Auch auf die Verschiedenheit der Aggregatzust"ande der Minerale kann man sich nicht berufen, wo es sich blo"s um Ein Mineral handelt, und jedenfalls, wenn in einem Kubikzentimeter vielleicht 50 verschiedene Formen gefunden werden, k"onnten "au"sere Ursachen nicht der Grund der verschiedenen Struktur, "`Aggregatzust"ande"' eines und desselben Minerals sein: aus dem einfachen Grunde, weil hier ein und dieselbe Ursache auf ein und dieselbe Substanz einwirkt, auch die vorhandenen Formen nicht etwa als Stufenfolgen der Kristallitenbildung angesehen werden k"onnten, denn sie find nahezu alle gleich entwickelt. Was aber schlie"slich den Aufschlag gibt, ist eben die Tatsache, dass kein Forscher meine Formen f"ur Kristalliten wird erkl"aren k"onnen, alles find hier Kurven, nirgends Winkel und selten gerade Linien. Jedenfalls wird kein Forscher zugeben, dass man mit einem einzigen zweifelhaften Mineral, welches "uberdies in allen seinen Erscheinungen nach der eigenen Beschreibung grundverschieden von meinen Formen ist, welches (ich fasse hier die Unterschiede noch einmal zusammen) andere Umrisse zeigt, n"amlich geradlinige Umrisse, statt Kreisdurchschnitte, welches ferner Spr"unge statt Zellenw"ande und Querw"ande (vergl. insbesondere Taf. 9, Taf. 11 Fig. 1 meines Werkes) --- welches S"aulen hat, die nicht radial geordnet find, statt der streng radial geordneten R"ohren meiner Formen --- welches Glaseinschl"usse ohne allen regelm"a"sigen Abstand hat (solche hat der Verfasser an seinem Mineral nicht wahrgenommen, sonst w"urde er es sicher gesagt haben), w"ahrend meine Formen solche zeigen ---: kein Forscher, sage ich, wird zugeben, dass man mit solchen Beobachtungen und Tatsachen erkl"aren und daher auch nicht, dass man mit solchen widerlegen kann.

Nun ich hoffe von der deutschen Wissenschaft, dass sie sich durch solche Gr"unde von der wirklich gr"undlichen Pr"ufung, welche die Sache nach meiner bisherigen Arbeit doch sicher verdient, nicht abschrecken lassen wird. Es ist wahrhaftig viel geringeren Objekten in der Mikrogeologie und Mineralogie so viel Ehre angetan und M"uhe zugewendet worden: ja man darf sagen, bis zur Gedankenlosigkeit oder wenigstens soweit, dass sich "uber die Beobachtungen selbst nichts mehr denken lie"s. In den Meteoriten und speziell Chondriten aber ist uns das Gestein erhalten, welches "uber die Planetenbildung und daher auch "uber die Bildung unserer Erde den einzig sicheren Aufschluss gibt. Dass ihre Untersuchung eine h"ochst notd"urftige war, das zeigt schon die Vergleichung des bisher Ver"offentlichten mit meinen Tafeln.

Der "au"sere Grund liegt wohl in der Seltenheit, der Kostbarkeit des Materials. Doch die Sparsamkeit in der Wissenschaft hat ihre Grenzen; l"asst man die Meteoriten, so wie sie heute in den Sammlungen liegen, so find sie ein toter Schatz. Auch ist nicht zu f"urchten, dass sie ausgehen; es fallen immer wieder.

Ist nun auch allerdings jeder Fall ein Unikum, so ist sein Wert eben doch auch wieder ein relativer, ein Wert nur, wenn man wei"s, was man hat. Man opfere nur, wie ich es aus privaten Mitteln getan habe, etwas, und die Sache wird bald entschieden sein, wer Recht hat, ich, der gesehen hat, oder Herr Rzehak, der nichts gesehen hat.

Die Entgegnungen auf die zoologischen Einw"urfe "uberlasse ich meinem Freunde Dr. Weinland.

Ich erlaube mir aber nur einen Satz des Herrn Rzehak herauszuziehen.

"`Undeutliche Gewirre kleiner Kristallleistchen h"alt Herr Dr. Hahn f"ur Nadelger"uste von Spongien."'

Darunter versteht der Herr Verfasser wohl die Abbildungen Taf. 8 meines Werkes. Gerade durch diese wurde ein Zoologe von bestem Namen "uberzeugt --- denn was Herr Rzehak in meinen Abbildungen nur als N"adelchen sieht, hat noch eine Struktur, und zwar eine sehr gute.

Jede Nadel hat einen scharf geschnittenen Hohlraum wie die Spongiennadeln. Ich fetzte diese Form unter die Abbildungen in der gewiss berechtigten Voraussetzung, dass sie von anderen Forschern, besonders von solchen, welche eine Kritik schreiben wollen (falls sie nicht die Objekte sehen wollen), im Gestein aufgesucht w"urden.

Die Schlussfolgerung, dass die Formen, wenn sie wirklich unsere terrestrischen Organismen gleich sein sollen, unter gleichen Voraussetzungen sich aufgebaut haben m"ussten, was augenscheinlich nicht der Fall sei, ist eben eine viel zu allgemeine Hypothese.

Hier entscheidet in erster Linie die Tatsache und wenn diese da ist, muss das Gesetz eingeschr"ankt werden. Doch ist auch der Satz des Herrn Verfassers selbst unrichtig. Was hei"st in der Natur "`gleiche Voraussetzung"'?

Wir haben \emph{Kalamiten} im Kohlengebirge und lebende, und doch find gewiss hier nach geologischen Begriffen wenigstens nicht dieselben Voraussetzungen vorhanden, aber dieselbe Sache nur in anderem Ma"sstab. Aber die Formen der Meteoriten find auch nur in ihrer allgemeinen Anordnung den terrestrischen gleich. Im gro"sen Verh"altnis z. B. unterscheidet sie sich sehr wesentlich: und das mag nun eben auf verschiedenen Voraussetzungen (Ursachen) ausgelegt werden. Dann haben wir die Ursache der Übereinstimmung, sowie der Unterscheidung.

Solche S"atze, sage ich, in der Allgemeinheit, wie sie der Herr Verfasser aufstellt, entscheiden nichts. W"urden sie aber doch in dieser schrankenlosen Allgemeinheit gelten, so kann ich mit demselben Rechte den Ausf"uhrungen des Herrn Verfassers folgenden Satz entgegenstellen:

"`Wenn die Chondrite, wie allgemein zugegeben, aus Enstatit und Olivin bestehen, so m"ussen, wenn sie nichts als Minerale find, unsere terrestrischen Olivine und Enstatite dieselben Strukturen zeigen, wie die meteoritischen, was bis jetzt wenigstens keineswegs zutrifft."'

Es l"age also hier f"ur zwei ganz verschiedene Tatsachen (Wirkungen) ein und dieselbe Ursache vor und da dies nicht m"oglich ist, schlie"se ich, und ich glaube mit demselben oder noch besseren Recht, als der Herr Verfasser, auf eine weitere Ursache der Bildung von solchen, welche au"serhalb des Minerals liegt, welche die organische ist.

Was die allgemeinen S"atze "uber die Art der Sch"opfung und insbesondere der Meteoritensch"opfung betrifft, so find diese erst zu er"ortern, wenn die Vorfrage ob Organismen (?) entschieden ist. Diese l"asst sich aber nicht mit einem (?) Kristall entscheiden, wenigstens kann dieser (?) Kristall, welchen "uberdies kein Dritter sehen, also sich auch nicht entscheiden kann, ob er auch das wirklich zeigt, was der Herr Verfasser von ihm sagt, etwas gegen meine photographischen Tatsachen beweisen. Zieht man aber diesen (?) Kristall, und zwar mit Recht von der Rechnung des Herrn Verfassers ab, so bleibt von seiner ganzen Ausf"uhrung nichts "ubrig, als allgemeine S"atze, deren Anwendbarkeit sehr fraglich ist, weil wir eben "uber die "`Voraussetzungen"' welche wir gleich nennen, sehr im Unklaren find.

Ich kann gegen die Ausf"uhrungen des Herrn Verfassers mich kurz auf die besten Bearbeitungen der Kristallite von Vogelsang, herausgegeben von Zirkel (Bonn 1875) berufen. Dieser gr"undliche Forscher hat merkw"urdige Formen abgebildet, welche man mit den meteoritischen vergleichen k"onnte, wenn, was er ausdr"ucklich hervorhebt, eine einzige da w"are, welche Struktur zeigte.

Eine solche ist nicht da. Dadurch unterscheiden sich Kristalliten von Organismen.

Was der Herr Verfasser etwa f"ur sich anf"uhren k"onnte, w"are die Abbildung Vogelsangs in dessen Philosophie der Geologie Taf. V, Mikrolithen-Konkretionen in gew"ohnlichem gr"unem Glase.

Allein die gro"sen und gerade erheblichsten Unterschiede treten auch sofort hervor --- keine W"ande --- regellos gelagerte Einschl"usse. --- Man nehme auch das Polarisationsmikroskop dazu, und niemand wird meine Korallen Taf. 8, 9, 11, Fig. 1 mit irgendeinem stengligen Mineral-Aggregat in Zusammenhang bringen.

Immerhin bleibt es freilich der n"achste Einwurf, dass die 6-8 Minerale, welche unsere s"amtlichen irdischen Gesteine konstituieren, nicht nur selbst, freilich nur oberfl"achlich betrachtet, sehr verschiedene Bilder zeigen, sondern auch in ihren Aggregatzust"anden zu den verschiedensten Formen f"uhren. Wer aber wirklich einen Gegenbeweis erbringen will, darf sich auch mit solchen allgemeinen S"atzen nicht begn"ugen: es w"urde dies einfach zuviel beweisen; allen Petrefakten w"urden damit wieder in das vierte Reich der Naturspiele zur"uckgef"uhrt. Die Entscheidung ist also blo"s im einzelnen Fall m"oglich. Aber es muss zun"achst in Betracht und abgezogen werden, dass jede Versteinerung notwendig zugleich die Eigenschaft des Minerals, in welches sie verwandelt ist, also auch feine Strukturform noch neben der urspr"unglichen organischen Struktur zeigen muss. Deshalb find Mineralerscheinungen kein Gegenbeweis gegen organischen Ursprung. Dieser Beweis w"urde wie gesagt dahin f"uhren, dass es gar keinen Petrefakten gebe. Die Frage ist nur, ob die besondere Strukturform neben der bekannten des Minerals aus demselben Mineral erkl"art werden k"onne? --- In dieser Beziehung behaupte ich von der meteoritischen Form, wenn sie genau beobachtet wird, dass dies nicht m"oglich ist, au"ser man verzichtet auf wissenschaftlich genaue Feststellung der Merkmale oder man beweist mit dem was erst bewiesen werden soll.

So kann ich dem weiteren Fortgang der Sache ruhig zusehen, die Frage ist nur, ob unsere Forscher wirklich und gewissenhaft sich die M"uhe nehmen wollen, die Sache zu pr"ufen, was ich doch nach den Vorarbeiten hoffen darf.

Dr. Otto Hahn.
\end{document}
