\documentclass[a4paper, 12pt, oneside]{article}
\usepackage[utf8]{inputenc}
\usepackage{fouriernc}
\usepackage{booktabs}
\setlength{\emergencystretch}{15pt}
\usepackage{fancyhdr}
\begin{document}
\begin{titlepage} % Suppresses headers and footers on the title page
	\centering % Centre everything on the title page
	\scshape % Use small caps for all text on the title page

	%------------------------------------------------
	%	Title
	%------------------------------------------------
	
	\rule{\textwidth}{1.6pt}\vspace*{-\baselineskip}\vspace*{2pt} % Thick horizontal rule
	\rule{\textwidth}{0.4pt} % Thin horizontal rule
	
	\vspace{1\baselineskip} % Whitespace above the title
	
	{\LARGE More About the Animal Remains in\\[0.25in] the Meteorites}
	
	\vspace{1\baselineskip} % Whitespace above the title

	\rule{\textwidth}{0.4pt}\vspace*{-\baselineskip}\vspace{3.2pt} % Thin horizontal rule
	\rule{\textwidth}{1.6pt} % Thick horizontal rule
	
	\vspace{1\baselineskip} % Whitespace after the title block
	
	%------------------------------------------------
	%	Subtitle
	%------------------------------------------------
	
	{Dr. David Friedrich Weinland} % Subtitle or further description
	
	\vspace*{1\baselineskip} % Whitespace under the subtitle
	
    {\small Das Ausland, No. 26, Article 1} % Subtitle or further description
    
	%------------------------------------------------
	%	Editor(s)
	%------------------------------------------------
	
	\vspace{1\baselineskip}

	{\small\scshape Stuttgart --- June 27, 1881 \\ }
			
    \vspace*{\fill}

    Internet Archive Online Edition  % Publication year
	
	{\small Attribution NonCommercial ShareAlike 4.0 International } % Publisher
\end{titlepage}
\setlength{\parskip}{1mm plus1mm minus1mm}
\clearpage
\paragraph{}
The critical remarks by Mr. Anton Rzehak from Brünn, published in No. 20 of this journal, about the organisms of the meteorites prompts me to say more on this matter, since Mr. Rzehak explicitly refers to my article about the corals in the meteorites in No. 16 of this body.

It is highly understandable that, as soon as a ``stone'' is concerned, the mineralogist initially upholds his right to it and claims the interpretation of its origin as well as its form, to a larger or lesser extent, as his task. No one will deny him this, and as long as he comes to a clear, scientifically understandable explanation, everyone will gladly like to believe the same. But as soon as the mineralogical interpretation of a ``stone'' becomes very difficult, as is admitted of the chondrules in the meteorites by all sides, the danger of an artificial, forced interpretation is very near, while perhaps another scientific discipline could give a very natural, and the only correct, explanation. Let's think about the history of petrifact studies. After all, it was not so long ago that people tried to explain the fossilized remains of animals, precisely because they were stones, in all possible ways, even as ``natural spectacles'', but never in the most natural and correct way --- until zoology took the matter into its own hands and created paleontology and, as we know, not without violent initial contradictions. Just think of the ``unanimous laughter'' of the French Academy, appealed to by Mr. Rzehak, when at the beginning of this century Cuvier established fossilized elephants. It will be the same with the chondrite meteorites and their inclusions. Not ten years will pass before we will have a small universally recognized fauna of the meteorites. This may still seem like a venturesome statement today, but my peers, who have known me for twenty-five years, will probably know that I do not easily pronounce my conviction. --- But to the point.

Dr. Hahn's meteorite work, based on hundreds of meteorite cuts, stemming from eighteen different meteorites, declared by one of the foremost German authorities, Professor R. [?], as ``regardless of the interpretation one wants, an excellent work of great scientific value'', Mr. Rzehak from Brünn tries to briefly dismiss, referring to a French mineralogist who once also wrote about meteorites and, of course, Dr. Hahn the ``German savant'' who, although a universally recognized capable mineralogist and excellent microscopist, is not actually a professional expert in his profession, but could neither readily prove the insufficiency of Hahn's observations nor, especially, of his illustrations. Then Mr. Rzehak points to his own observations on a few meteorite cuts from the fall of Tieschitz in Moravia, in which he believes he has found all the material needed to declare the entire work of Dr. Hahn as \emph{ad absurdum}. ---

Certainly every expert first approached this work with great doubts. The matter came quite suddenly. Some of the forms depicted by Hahn had to have been immediately recognized by every connoisseur of the microscopic as typically organic animal structures, but their origin triggered a reminder to be cautious. Thus, as far as we know, no German researcher has dared to pronounce an unconditional positive or negative judgement, especially in public, merely their opinions of the work, and without viewing the objects themselves. ---

The above-mentioned notice in \emph{Das Ausland} about the corals in the meteorites was written by me when I had only studied a few, especially desirable, cuts. Since then, I have had at my disposal for months the rich meteorite collection of Dr. Hahn and I have not only had the opportunity to study the pieces pictured by him, but also a large number of new pieces, which are especially far-reaching for the zoologist. The fact that in the chondrite meteorites, some less, others more, we are dealing with a multitude of organic inclusions, and indeed from very different families and classes, of related animal detritus is beyond reproach. A brief compilation of the results from my previous studies, in which I characterize a number of genera and species and which will include some illustrations, is to be published in the \emph{Leopoldina} during the summer and is already in this academy. A larger work for the \emph{Acta} of the same academy, with detailed structural descriptions and drawings, is in preparation. I could refer to these two, but in our fast-paced times we do not like to be consoled with the future, so I allow myself to mention a few things here, but I expressly indicate that my position in the matter is completely impartial and that in my interpretation of the forms and results I do not feel in any way bound by the earlier interpretations of Dr. Hahn in his meteorite work or his conclusions, about which I have talked to Dr. Hahn and completely communicated the zoological treatment of his discovery. For me, from the outset, it was only a question of: are the structures in question organic forms, what kind are they in comparison with terrestrial ones, and what direct conclusions do their presence in the meteorites indicate about their origin?

Now several points:
\begin{enumerate}
\item The various chondrite meteorites are not equally rich in their organic structures, some consist of two-thirds or more of them. As a rule, there are smaller or larger fragments and usually only after working through a large number of cuts does one find a whole one amongst the different structures, just as it is known even with rare terrestrial petrifacts. ``Magnificent specimens of organisms'', as Mr. Rzehak looks for them in his first and only Moravian cuts, are unfortunately quite rare. We only have a dozen of them in six hundred cuts. By such, I mean, above all, those forms in which a large part of the external contours of the animal organism come into view simultaneously with the internal structure. For example, I have found a sponge shape, and precisely this one, in a number of pieces where not only the outer shape, which is flat-bottomed, rounded-off and lobe-like, but also, by accidental fortunate cuts, ones where from above the porous covering layer of the sponge and generally the mesh skeleton of the gastrovascular system filling the sponge is perfectly preserved, as well as in any terrestrial petrifact. I intend to call these forms --- with the permission of Mr. Rzehak, who does not seem to particularly like my genus name \emph{Hahnia} --- \emph{Pectiscus}. Other sponge forms, likewise in large numbers but with different, finer covering layers and other very strange star-shaped mesh gastrovascular systems, I propose to leave the name \emph{Urania} that Dr. Hahn originally created for this form, of course when he used to think that all these structures were plants, for which Mr. Rzehak takes so much offense, but perhaps my dear friend Dr. Hahn would rather send his apologies when he remembers that at the beginning of this century the sponges were declared as plants by many proficient researchers. I would like to add here that for Dr. Hahn, as he expressly explains in his book, the zoological classification of his forms was not the main concern and could not be, because he is not an expert in zoology. His only concern was to prove that there are organic formations in the meteorites and this is, and will remain, his great and meaningful merit, though with his zoological interpretations, especially that of the crinoids, etc., I cannot follow everywhere he wants to go.
\item It is by no means a single-handed bargain, as Mr. Rzehak seems to assume, that the explanation of these fibrous or columnar structures, so well described by Director Gümbel, and which Mr. Rzehak also finds in his Moravian meteorites and even observed in a questionable feldspar, whose transverse partitions he declares to be ``transverse fissures'' (but our instruments do not show fissures, but distinct, bodily partition walls), and besides a large number of additional quite different structures which have not the least to do with fibers (i.e., in reality tubes arranged in parallel), e.g. besides the previously mentioned sponge forms \emph{Pectiscus} and \emph{Urania} there is another hahnia-pectinate structure that will probably belong to the Foraminifera and reminds us of the \emph{Carpenteria rhaphidodendron} of Möbius; there are also faceted spheres that are regularly stacked upon each other's silicic joists and they themselves are hollow, have little holes, and that I can only compare with those delicate radiolarian skeletons depicted by Haeckel in his beautiful works. (Dr. Hahn had placed them as crinoids up till now; regarding the other so-called crinoids of Hahn, which are especially troublesome for our Mr. Rzehak, I will give a more detailed presentation in the relevant place). Further, there are other forms, also probably belonging to the radiolarians, whose silicic joists on the periphery merge into a network of meshes, and again other shield shaped ones whose description without illustrations would not give a clear conception, etc..
\item The first impression obtained in the measurement of these meteorite forms is one of an extraordinary smallness, as Hahn has pointed out and I noted in my first article in \emph{Das Ausland}. But now that a greater number of forms are recognized as foraminifera and radiolarians, whose size agrees quite well with that of terrestrial forms, only the corals of the meteorites remain as unusually small structures. But even with these, the relationship is not so extraordinary. Terrestrial corals are known with calyxes of 1 mm diameters, yes even 0.5 mm, while those of the meteorites measure up to 0.1 mm. Likewise, there are also microscopic species of terrestrial sponges. If we also consider that we mostly work with thin sections of these meteorites, it is then understandable that larger shapes are not likely to be observed, even as fragments, in the countless structures we observe in the cuts.
\item A big misconception would be the hypothesis that I have recently encountered in a letter from an eminent writer, and that may also be held by others who are not familiar with the composition of the chondrite meteorites, saying that these organic forms might be the remains of lower animals that arose on the surface during their course through space. Naturally this is not the case. Rather, these structures are inclusions in the meteorites. They are petrifacts, nothing else, and the chondrite meteorites themselves seem to us to be merely the primary petrifact rock debris of a foreign heavenly body, though certainly interesting enough as such. ---
\end{enumerate}
\paragraph{}
We kindly ask Mr. Rzehak from Brünn, before he can continue to be heard on the matter, either for a gracious inspection of our cuts themselves or for further cuts and then more microscopy, as Hahn and I have been doing for months. Then who knows, maybe in one way or another he will become an advocate of Hahn's discovery, as has recently become of a well-known South German mineralogist and paleontologist at my microscope.
\rule{\textwidth}{1.6pt}\vspace*{-\baselineskip}\vspace*{2pt} % Thick horizontal rule
\rule{\textwidth}{0.4pt} % Thin horizontal rule
\paragraph{}
Because the issue discussed above is of tremendous importance to modern science, and because of the lively discussion it arouses in the participating circles, the editorial board believes that Dr. Weinland's preceding explanations should be immediately followed by Dr. Otto Hahn's comments. Dr. Otto Hahn writes:

In No. 20 of \emph{Das Ausland} Mr. Anton Rzehak from Brünn goes against the ``Organisms of the Meteorite''.

His evidence is essentially the following:
\begin{enumerate}
\item The Paris Academy has not accepted the case.
\item Hahn's conditions for organic nature are not correctly stated, because two of the five characteristics given are not themselves evidence of an organism.
\item There are --- and here Mr. Rzehak refers to a mineral with a question mark (feldspar (?)) that has quite clear columnar constructions but are admittedly not radially arranged --- also tubular formations in the mineral kingdom, thus he concludes that the tubes in the chondrites are not necessarily of organic origin.
\item Enstatite and tourmaline have transverse fissures that can easily be confused with the transverse partition walls of organisms.
\item In the ``feldspar (?)'' mineral Mr. Rzehak sees several inclusions with longitudinally arranged rows: he therefore concludes that ``obviously'' such inclusions in the chondrite minerals have been mistaken for ``perforations''.
\item Hailstones also occur that possess structures similar to that of the chondrites. (Gümbel.)
\end{enumerate}
\paragraph{}
What is further argued is only criticism of conclusions, which will I leave aside because if my facts are correct then this criticism falls by itself.

As far as the authority of the Paris Academy is concerned, I only note that it is the same academy which, for nine years after the publication of Chladni's book on the cosmic origin of meteorites, declared the proposition of falling meteorites as madness but then, after all, it was only after nine years that a post office worker convinced himself of the incorrectness of their previous opinion. Their consolation at the time was the following phrases: ``the fool believes'', ``the half-educated concludes'', ``the educated verifies'', certainly light consolation for such errors (Quenstedt, \emph{Klar und Wahr} p. 287).

When Mr. Rzehak summons the judgement of the \emph{Comptes Rendus}, I must add that the member of the Paris Academy, Mr. Daubrée (not Dumas), who accepted my work replied to me that he had obtained similar forms by melting the forms found in the chondrites; however, at my request for information on such a melt product I received neither an answer nor such a product: proceedings that do not suggest the correctness of a claim.

In his book \emph{Experimental Geology}, p. 386, Mr. Daubrée depicts the Knyahynia meteorite, though not very accurately. That the inclusions have structures, he has overlooked, for the simple reason that all his investigations begin with powders and the melting of stone.

Even the Academy of 1800 still had hundreds of ``physical and moral arguments'' against the cosmic origin of meteorites, a view which if repeated today would have no success other than that of laughter.

However, Mr. Rzehak has ``physical and moral arguments'' against my work, which I will now discuss in more detail.

Above all, he contests my definition of the organic by not allowing two features of my notion, namely ``closed form'' and ``recurring form'', to be sufficient in themselves to prove the existence of an organism. But since I called for five related traits as proof of an organic being, I myself declare these two characteristics as insufficient proof by themselves: as an argument against me, this is not truthful.

With only statements three, four, and five the author of the criticism wants to explain the structure of the chondrites from minerals, provided that he apologizes to Gümbel.

Mr. Rzehak does not think it necessary to address the negative proof, that they are not mineral formations, only the positive that deals with real organisms: nor do my thirty-two photographic tables exist for him. That they possess significance in themselves, I appeal to the judgement of the foremost authority in the field of mineral structure, which is as follows: ``regardless of the interpretation one wants, your book must, in any case, be regarded as an excellent work on the structure of the meteorites and whose tables are of the greatest scientific value.''

And what is the evidence of Mr. Rzehak? One (!) mineral, which he cannot even determine --- evidence that either the mineral is uncertain, and therefore not evidence, or that Mr. Rzehak is not a mineralogist. This suggests that his mineral appears unique, although a hollow form in feldspar (in the process of corrosion) is a very common phenomenon. It is not necessary to have this shadowy crystal brought up at all, as Mr. Rzehak could have directly and briefly referred to this fact, but of course I would have then pointed out to him the difference between mineral and organism. About this (?) mineral Mr. Rzehak gives no picture for readers to see and judge for themselves.

On the whole, this is not addressed. He concludes \emph{a priori} where facts exist, apologizing for minerals that no one can see and compare, and making light of things that are obviously unknown.

If I were to concede any verdicts to Mr. Rzehak, he would first have to assure me that he knew my material or at least saw as much material as I have. But to the point!

My proof, first of all, is a negative one, i.e. proof that the mineral structures are not possible: and a positive one, that the forms of the meteorites are in accordance with recognized organic forms.

The first argument, the negative one, is (as I said) completely ignored in the critique. Above all, I would have expected a refutation of this part, since it is accessible to anyone. I refer here to my book of meteorites (chondrites) p. 20, please read it.

I would like to hear only one question answered by Mr. Rzehak, if he is a mineralogist: how is it possible that one or two minerals, as is commonly assumed that the chondrites are composed of, that are in the same stone (of some five hundredweights), that is, born and formed under the same conditions, display all the hundreds of structural forms that I have depicted in my work? And now multiply these structures by twenty-five.

Mr. Rzehak does not give an answer to this question, which I had already raised in my book: he is content to quote Gümbel, who believed that he had found structures similar to those of the chondrites in ice (hailstones). --- It would indeed cause a great stir if ice and enstatite crystallites were similar. That there are seemingly columnar structures in ice and many mineral aggregates is certain, except the difference is that in the chondrites there are not only fracture (optical) lines, but truly substantial walls formed by a second mineral; these ``columns'' are not all in a mess like in the (?) mineral of Mr. Rzehak, but quite regularly arranged, and indeed eccentrically and not concentrically, and furthermore the parts do not form a sphere, but rather a flat sheet of tubes. The crux of my demonstration, the key to my position, is the frequent large and small structures, the regularity of which absolutely excludes the supposition of natural inclusions.

Therefore, I have given a number of such under high magnification, like Table 9 and 15; I have also supplemented the text with what I could not show, at least through the photographs, at such high magnification.

Against these photographic images of the structure of the chondrites, the author cites and describes his observation, as stated above, of a mineral with a question mark; he thinks it is feldspar. In the mineral (which the author does not know himself) he has observed a ``columnar construction''. But first of all, you will probably remember, he did not find curved columns, as found in my forms, but rectilinear outlines. Just as well, and far easier, he could have summoned basalt columns as a counter proof.

The fact that my structures are curved tubes is either overlooked or concealed by Mr. Rzehak, but both are necessary to mention, with the latter doubly so, because my book, as the author himself says, is only in a few hands, while his criticism reaches many hands. Now, tubes!

To refute my notion would require that he demonstrate tubular structure in his (?) mineral. That there are crystal aggregates with rectilinear outlines requires no need for a mineral with a question mark: everyone knows this, even the layman. But that there are minerals (and not aggregates) that consist entirely of curved tubes, I have neither read nor seen.

A mere mineral has no structure at all, it can only reproduce a kind of mechanical outgrowth or chemical dissociation pattern structure, to which it recursively unites with the original mineral. So the observation about the feldspar in question does not apply here at all.

That which distinguishes the crystal columns from the curved tubes of the chondrites, I mentioned in my book: there are substances that form the walls of the tube, and a filler, two minerals that constitute the tubes while crystal columns consist of only one mineral, and visible only as cracks (optical lines) that become noticable. Further, as Mr. Rzehak admits, these ``columns'' are not radially arranged like those in the chondrites but chaotic, and it only takes one glance into a polarization microscope to demonstrate the difference between the two formations in full light. Moreover, as I stated above, there are fan-like tubes: and formed purely as a series of tubes adjacent to each other, deposited strictly (ec-)centrically. Of course, it is easy to ``demonstrate'' with such objects and facts as Mr. Rzehak, being certain that the reader will see neither the object of attacked nor that of the attacker; even an expert reads such things in good faith, easily overlooking the differences because he does not even have the book of the rival in front of him. Such argument is either unforgivably superficial --- or --- (if knowingly) dishonest.

Thirdly, in order to explain the finer structures, the ``favositoid'' channels mineralogically, or more correctly, to establish me in this direction as delusional, he summons the glass inclusions in the (?) mineral that gives an impression of a transverse channel and, where they line together, that of a hollow space.

As I presented and said in my book, the channels of the meteorite (\emph{Favosites}) are in totally equal sections, the glass inclusions are not, and here I want to add that they are in cross-sections not as points, but are present as clear transverse channels, hence not a not disseminated mineral (spot), but truly are quite undoubted tubes (germination channels of \emph{Favosites}). Here and after one can no longer speak of round glass inclusions as counter evidence to the fact. Not yet a researcher who has seen my objects has made the objection that what I declare as germination channels (perforations) are mere inclusions.

Here I must go even further and point out the biggest mistake of all criticism dealing with external perceptions: it exists therein, that one criticizes observations of third parties before one has seen the observation-material of the objected.

And to return to the present case, I at the least insist that, from hereon, controlled cuts should be performed on Knyahinya.

I can assure the author that I have already seen hundreds and thousands of glass inclusions, but no rock has remotely demonstrated what I have observed in the chondrites. Here, at a magnification of 1000x, there are not found magnetite grains as often occur in meteorite rocks, nor arbitrarily shaped glass inclusions, but circular, sometimes elliptical shaped surfaces with a wall and at least a darker colored mass between the circle and its surroundings; moreover, this circle often lies in a depression (which one can really see in Table 15): the ``perforations'' are found only in tubes, and finally, the entrance-wall is pierced laterally by the channels, which are symmetric and equidistant to those which are seen as points in the cross-section. These lateral intersections are quite clear in form, Table 8 at 300x magnification. This is something other than an infilling or inclusion.

In the fourth place, the gentleman author concludes with an explanation of the transverse partition walls. Here too his criticism is incorrect.

It is well-known to me. In my book, however, I discuss this objection, both as it regards the explanation of the tubes and lamellas from sheet breakage and as it concerns the transverse partition walls from transverse fissures, and point out that both the sheet breakage and the transverse fissures are merely optical phenomena, while the cell partition walls of the organisms and especially the transverse partition walls in my forms are built of special substances. Therefore, to show an image of simple breakage and partitioning, I have depicted a terrestrial enstatite (Texas) that is a mere mineral whose fractures appear as black lines.

However, the enstatite of the Bishopsville meteorite is a pure enstatite mineral and coincides with that of Texas in Table 1: Figure 2 (i.e. a meteoritic enstatite with a terrestrial enstatite) so perfectly that the images cannot be distinguished. If meteoritic enstatite, where it exists only as a mineral, has the same structure as the terrestrial one, it follows that if the meteoritic minerals have completely different structures, then the latter must have a special cause (not located in the mineral).

Here I must lead with a fact that has long been known.

When an organism is ``petrified'', a mineral takes the place of organic material. It may leave some of the original substance behind, e.g. the silicic scaffolding of sponges. Yet this does not come into consideration here. Usually all of the substance is recast, or at least the cavities are filled afresh. The transforming mineral is a mineral and remains so, and as such it has its properties: it is only capable of crowding the place of the original organism, whose outermost contours remain preserved, while the entire form is filled by the mineral. Such a form is demonstrated, e.g. calcite with its three sheet breakages, at the site of the \emph{Cidaris} spikes, which Quenstedt indicates in \emph{Epochs of Nature}, p. 558. The \emph{Cidaris} spike is a pure calcite substance, though its contours are completely maintained, so that nobody would suspect it as merely calcite with sheet breakages. This is partly the type of petrification in the chondrite organisms. Externally enstatite, internally olivine. Also however, the structures, where they are preserved, are merely filled with the mineral and thus they have all its ordinary physical properties. Hence, by necessity the mineral properties (mineral structures) become the remnants of organic matter and structure and on account of this will always be so: if an opponent merely mentions just the former --- the mineralogical phenomena --- and claims it as merely a mineral, he is at least right for the moment. But as soon as one demands from him an explanation for the truly organic structures, his skill will forsake him. Of course, he likes to use common expressions like ``reminds one of'', ``is analogous, though not identical'', ``indicates relationship'', and the like. Such expressions have legitimacy where an analogy genuinely exist. But even an analogy has its scientific limit, otherwise a pigeon could after all ``remind one'' of a roof tile. Here then is exactly where the most exact observation and comparison of the characteristics must occur. Regarding the meteorite forms, however, only terrestrial enstatite and olivine can be permitted as an analogy, but by no means ice or any feldspar, etc.: strictly speaking, once enstatite and olivine structures are found to be present in a meteorite, as in a terrestrial occurrence, a reference to other minerals ceases to apply: here the analogy proves itself immediately, that one is not dealing with mineral structure. Nor can we summon the diversity of the aggregate states of minerals where there is only a single mineral, especially if fifty different forms are found in one cubic centimeter, as external causes may not be the reason for the different structures, that is, ``aggregate states'' of one and the same mineral: for the simple reason that one and the same cause acts on one and the same substance and the forms present cannot be regarded as a hierarchy of crystallite formation because they are almost all equally developed. But what gives the final impact is the fact that no researcher is able to explain my forms as crystallites, everything here is curves, nowhere angular and straight lines. At any rate, no researcher will admit that with a single dubious mineral, which in all its manifestations is fundamentally different from my forms, that (I will summarize the differences here again) displays different outlines, namely rectilinear outlines instead of circular averages, and has fissures rather than cell walls and transverse partitions (see in particular Table 9 and Table 11: Figure 1 of my work) --- which contains columns that are not radially ordered instead of the strictly radial arrangement found in my tube forms --- which contains glass inclusions that lack a constant spacing (this is not perceived in the author's mineral, otherwise he would have said so), whereas my forms demonstrate such --- no researcher, I say, will admit that these observations and facts explain and hence refute such.

I hope for German science that it will not be deterred by such reasons from a truly thorough examination, which is surely needed after my previous work. Indeed, much lesser objects in microgeology and mineralogy have been done with much more honor and effort: one may even say, to the point of thoughtlessness or at least to the point where nothing is left to be thought about other than the observations themselves. In the meteorites, and specifically the chondrites, rock is preserved that provides the only certain information about planet formation and also the formation of the Earth. That this investigation was a highly needed one is evident by comparing what was published previously with my tables.

The external reason probably lies in the rarity and preciousness of the material. But thrift in science has its limits; if the meteorites are left as they are in the collections today, they are a dead treasure. Nor should one fear that they will run out; they will always fall again.

However, if each case is unique, then its value is also relative, a value that is only known by what one has. One simply sacrifices, as I have done through private means, and the matter will soon be decided: who is right, I, who has seen, or Mr. Rzehak, who has seen nothing.

I leave the reply to the zoological objections to my friend Dr. Weinland.

I allow myself to extract but one sentence from Mr. Rzehak.

``Dr. Hahn considers indistinct tangles of small crystal bands to be needle spicules of sponges.''

By this, the author probably refers to the pictures of Table 8 of my work. It was precisely through these that a zoologist of the best name was convinced --- because what Mr. Rzehak sees in my pictures only as needles still has structure, and indeed one of high quality.

Each needle has a sharply cut cavity like the needle sponge. I put this form among the figures in the rock with the justified stipulation that it would be visited by other researchers, particularly those who want to write a review (if they are unable to see the objects).

The deduction that the forms, if they are genuinely similar to our terrestrial organisms, must have been built up under identical conditions, which is obviously not the case, is a much too general hypothesis.

First, a line of facts decides and, provided it does, a law must then be limited. But the sentence of the gentleman author himself is incorrect. What does ``identical conditions'' mean in nature?

In the coal rock masses we have \emph{Calamites} living, here in geological terms, certainly not with the same conditions present, but the same object only in other norms. However, the forms in the meteorites are similar only in their general design to the terrestrial ones. In a large proportion, for instance, it is very different: and this might be interpreted as different conditions (causes). Then we have the cause of the agreement, as well as the distinction.

Such sentences, I say to the general public, as the author puts it, decide nothing. But if they are to be effective in this barrierless general public, then I can with the same right stand up to the author's following statement:

``If the chondrites, as generally admitted, consist of enstatite and olivine, and if they are nothing but minerals, then our terrestrial olivine and enstatite must show the same structures as the meteoritic ones, which is nowadays by no means true.''

Here there are two very different facts (effects) with the same cause, and since this is not possible, I conclude and believe, with the same or even better right than the author, in another cause of formation that is outside the mineral, which is the organic one.

Regarding the general propositions concerning the nature of creation, specifically meteorite creation, these should only be discussed once the preliminary question of whether there are questionable organisms has been decided. But this cannot be done with a (?) crystal, at least this (?) crystal, which no third party sees, cannot decide whether or not the author is really seeing what he says as something against my photographic facts. But if one subtracts this (?) crystal, and rightly so from the account of the gentleman author, nothing remains of his entire performance, only general propositions whose applicability are quite questionable, because we are very unclear about the ``conditions'' which we also describe.

Contrary to the remarks of the author, I can briefly point to the fine arrangements of crystallites of Vogelsang, published by Zirkel (Bonn 1875). This thorough researcher has depicted strange forms that could be compared to the meteoritic ones if, as he expressly points out, there was a single one that showed structure.

Here there is not such. As a result, crystallites are distinguishable from organisms.

For instance, what the author might cite for himself would be the depiction of Vogelsang in his \emph{Philosophy of Geology}, Table 5, microlith-concretions in ordinary green glass.

But the great and most significant differences emerge immediately --- no walls --- randomly stored inclusions. --- Include the polarizing microscope and no one will associate my corals of Table 8 and 9 and Table 11: Figure 1 with any columnar mineral aggregate.

Though it remains as the next objection, that the six to eight minerals which constitute all our terrestrial rocks not only show very different images themselves, of course only superficially, but also lead to the most diverse forms in their aggregate states. But whoever really wants to prove otherwise cannot be content with such general sentences: it would clearly justify too much; all petrifacts would be brought back into the fourth realm of natural spectacles. The decision is therefore only possible in individual cases. But it must first be considered and deduced that every petrification must simultaneously display the properties of the mineral into which it has been transformed, i.e. its fine structural form, in addition to the original organic structure. Thus, mineral phenomena are not counter-evidence against an organic origin. Such evidence, as I said, would lead to the fact that there would be no petrifacts at all. The only question is whether a particular structural form of this same mineral can be explained by the known minerals? --- In this respect I claim of the meteoritic forms, if they are closely observed, that it is not possible unless one refrains from a scientifically accurate determination of the characteristics or one proves with what is to be proven first.

I am still watching the progress of the matter, the only question is whether our researchers truly and conscientiously desire to take the trouble to examine the matter, which I can hope for after this preparatory work.

Dr. Otto Hahn
\end{document}
