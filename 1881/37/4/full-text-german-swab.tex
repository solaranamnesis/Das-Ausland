\documentclass[a4paper, 12pt, oneside]{article}
\usepackage[utf8]{inputenc}
\usepackage[T1]{fontenc}
\usepackage[ngerman]{babel}
\usepackage{yfonts}
%\usepackage{fbb} %Derived from Cardo, provides a Bembo-like font family in otf and pfb format plus LaTeX font support files
\usepackage{booktabs}
\setlength{\emergencystretch}{15pt}
\usepackage{fancyhdr}
\usepackage{microtype}
\begin{document}
\swabfamily
\begin{titlepage} % Suppresses headers and footers on the title page
	\centering % Centre everything on the title page
	\scshape % Use small caps for all text on the title page

	%------------------------------------------------
	%	Title
	%------------------------------------------------
	
	\rule{\textwidth}{1.6pt}\vspace*{-\baselineskip}\vspace*{2pt} % Thick horizontal rule
	\rule{\textwidth}{0.4pt} % Thin horizontal rule
	
	\vspace{1\baselineskip} % Whitespace above the title
	
	{\LARGE Nochmals die}
	
	{\LARGE "`Organismen der Meteorite"'}
	
	\vspace{1\baselineskip} % Whitespace above the title

	\rule{\textwidth}{0.4pt}\vspace*{-\baselineskip}\vspace{3.2pt} % Thin horizontal rule
	\rule{\textwidth}{1.6pt} % Thick horizontal rule
	
	\vspace{1\baselineskip} % Whitespace after the title block
	
	%------------------------------------------------
	%	Subtitle
	%------------------------------------------------
	
	{Anton Rzehak} % Subtitle or further description
	
	\vspace*{1\baselineskip} % Whitespace under the subtitle
	
    {\small Das Ausland, Nr. 37, Artikel 4} % Subtitle or further description
    
	%------------------------------------------------
	%	Editor(s)
	%------------------------------------------------
	
	\vspace{1\baselineskip}

	{\small\scshape Stuttgart --- 12. Sept. 1881}
			
    \vspace*{\fill}

    Internet Archive Online Edition  % Publication year
	
	{\small Namensnennung Nicht-kommerziell Weitergabe unter gleichen Bedingungen 4.0 International} % Publisher
\end{titlepage}
\setlength{\parskip}{1mm plus1mm minus1mm}
\clearpage
\paragraph{}
Ich wei"s nicht, ob es f"ur die Wissenschaft ein Vorteil ist, wenn die Vertreter derselben gewissen literarischen Erzeugnissen gegen"uber eine Gleichg"ultigkeit an den Tag legen, die sehr leicht missdeutet werden kann. "`\emph{Qui tacet, consentire videtur}"'; nach diesem Grundsatz schlie"st das gro"se Publikum und macht die k"uhnste Hypothese, wenn gegen dieselbe von ma"sgebender Seite keinerlei Einwendungen erhoben werden, ohne weiters zu einem Dogma. W"ahrend sich der Gelehrte mit einem entsprechenden Grade von Wahrscheinlichkeit begn"ugt, meint der Laie mit Fug und Recht nach Wahrheit fragen zu d"urfen; der ganze komplizierte Apparat der wissenschaftlichen T"atigkeit, das mannigfache Ineinander gehen und Zusammenwirken verschiedener Disziplinen ist ihm v"ollig fremd. Solche Errungenschaften, die ein allgemeines Interesse in Anspruch nehmen, werden bald im Publikum bekannt und sorglich in der f"ur jeden Gebildeten "`von heutzutage"' unentbehrlich gewordenen wissenschaftlichen Vorratskammer "`f"ur den Hausgebrauch"' aufbewahrt. Der Kontakt zwischen den Gelehrten und dem Publikum wird zumeist nur durch die Tagesjournalistik vermittelt; die Vermittler sind in der Regel nicht im stande, den Ma"sstab wissenschaftlicher Kritik selbst anzulegen, m"ussen aber doch trachten, dem im Publikum faktisch bestehenden Bed"urfnisse gerecht zu werden. Und so pfl"ucken sie denn hie und da vom B"aume der Wissenschaft eine Frucht und bieten sie zum Gen"usse dar, ohne R"ucksicht darauf, ob diese Frucht auch bereits reif und genie"sbar ist. Auf diese Art geschieht es, dass im Publikum verschiedene Ansichten, "uber welche die Gelehrten noch keineswegs einig sind, verbreitet und \emph{bona fide} als Tatsachen hingenommen werden. So ging es auch mit den "`Organismen der Meteorite"'; die "`Entdeckung"' des Herrn Dr. Hahn wurde in zahlreichen Zeitschriften ohne jegliche Kritik besprochen und scheint ganz popul"ar werden zu wollen, ehe sie noch von kompetenter Seite best"atigt oder widerlegt wird. Sowohl \emph{pro als contra} haben sich bisher nur wenige Stimmen erhoben, obwohl die Sache unleugbar eine tiefere Bedeutung f"ur die ganze monistische Weltanschauung besitzt. Die M"oglichkeit des Vorkommens organisierter Gebilde in Meteorsteinen ist durchaus nicht von vornherein ausgeschlossen, und wenn nun dieses Vorkommen nicht etwa nur mit Wahrscheinlichkeit, sondern mit Gewissheit behauptet wird, so tritt an den Fachmann die Verpflichtung heran, die Sache einer unparteiischen, vorurteilsfreien Kritik zu unterziehen. Wie kommt es jedoch, dass man sich allgemein scheut, in einer so interessanten Frage sein Urteil offen auszusprechen? Unwillk"urlich wird man dadurch an die "angstlichkeit erinnert, mit welcher die Gelehrten zu Anfang dieses Jahrhunderts den von Chladni "uber den Ursprung der Meteoriten geltend gemachten Ansichten auszuweichen suchten. Man behauptete damals, "`Chladni habe nur eine paradoxe Meinung so hingeworfen, und mit allen m"oglichen Scheingr"unden ausstaffiert, um, wenn sie von den Physikern ernstlich aufgenommen w"urde, sich "uber die lustig zu machen."' Vielleicht hegt man bez"uglich der Hahnschen "`Entdeckung"' "ahnliche Bef"urchtungen; glaubt man jedoch die Ausgeburten der Dilettanten-Gelehrsamkeit dadurch unsch"adlich zu machen, dass man sie ganz einfach ignoriert?

Bergrat Dr. Jenzsch, ein Vorl"aufer des Herrn Dr. Hahn, glaubt vor Jahren in Melaphyr- und Porphyrgesteinen Reste fossiler Organismen entdeckt zu haben; er verstieg sich freilich nicht bis zu den Korallen und Crinoiden, sondern sprach nur von vollkommen gut erhaltenen Algen, Infusorien und R"adertieren. J. G. Bornemann hat die Entdeckung des Dr. Jenzsch gelegentlich der Natur Forscherversammlung zu Dresden (1868) besprochen und nachgewiesen, "`dass sich unter allen angeblichen Tier- und Pflanzenresten nicht das geringste befand, was nicht auf nat"urliche Weise als eine anorganische Erscheinung und ein auf rein physikalischem Wege entstandenes Gebilde h"atte gedeutet werden m"ussen."' Darf man Bornemann vielleicht einen Vorwurf machen, dass er es nicht unter seiner W"urde hielt, die Ansichten des Bergrats Jenzsch zu pr"ufen? Gewiss nicht!

Dr. Hahn steht mit seiner Ansicht nicht mehr isoliert da; er hat in Dr. Weinland einen Verteidiger gefunden, welch letzterer wieder einen deutschen Pal"aontologen, dessen Name leider verschwiegen wurde, von der zoomorphen Natur der Chondren "uberzeugt haben will. Unter diesen Umst"anden ist es denn doch geboten, die unparteiische Pr"ufung der Sache in Angriff zu nehmen, und fordere ich hiermit alle jene dazu auf, welche Gelegenheit haben, D"unnschliffe von Chondriten zu untersuchen. Herrn Dr. Hahn braucht es bei dieser Aufforderung nicht bange zu werden; ist seine Ansicht eine richtige, so wird sie endlich, trotz aller Angriffe, auch als solche anerkannt werden.

Es handelt sich hier wesentlich nur um die Frage: "`Ist die Struktur der Chondren eine rein mineralogische oder nicht?"'

Die meisten Meteoritenkenner werden diese Frage wohl ohne weiteres in bejahendem Sinne beantworten; man muss indessen trachten, um nicht von Dr. Hahn der "`Oberfl"achlichkeit"' oder "`Unehrlichkeit"' geziehen zu werden, m"oglichst viele Gr"unde, welche f"ur die anorganische Struktur der Chondren beweisend sein k"onnen, beizubringen und "`schwarz auf wei"s"' darzulegen.

Die Eigent"umlichkeiten der Chondrite hat bereits G. Rose hervorgehoben, und wohl ein jeder, der dieselben zu studieren Gelegenheit gehabt hat, ist auf den Gedanken gekommen, dass die Bildungsweise derselben verschieden gewesen sein mag von den uns bekannten Bildungsweisen terrestrischer Gesteine. Die Analogie der letzteren mit den Chondriten ist trotz mancher "ahnlichkeiten doch nur eine unvollkommene. G"umbel erkl"art die Chondrite f"ur Tr"ummergesteine und Tschermak findet in der eigent"umlichen Struktur derselben gewisse Ankl"ange an die terrestrischen Tuffe; er denkt jedoch hierbei an eine Zerreibung starrer Massen und schlie"st die T"atigkeit des Wassers bei Bildung der Chondrite aus.

Nach der Ansicht des Herrn Dr. Hahn m"ussten die Chondrite rein klastische Gesteine sein, die in sehr ruhigem Wasser zur Ablagerung gelangt sind, nachdem "`nirgends abgerollte Formen oder Splitter"' vorhanden sind. Dennoch meint Dr. Hahn, "`dass das Gestein der Chondrite nicht etwa nach Art unserer Sedimentgesteine ein Schlamm war, in welchen die Tiere eingelagert wurden."' Die "`ganze Masse"' soll aus Organismen bestanden haben; dann bleibt es jedoch sehr r"atselhaft, an was die Crinoiden, Korallen und Schw"amme, deren Anwachstellen ja Herr Dr. Hahn ganz deutlich beobachtet hat, eigentlich befestigt waren??

Auf keinen Fall zeigen die Chondrite eine wesentlichere "ubereinstimmung mit den klastischen Gesteinen der Erdrinde. G"umbels Ansicht, nach welcher die Meteorite "`aus einer Art erstem Verschlackungsproze"s der Himmelsk"orper"' hervorgegangen sein sollen, scheint die einzig m"ogliche Deutung ihres eigent"umlichen Wesens zu sein. Daubrée hat bekanntlich "uber die Bildungsweise der Meteoriten sehr interessante synthetische Versuche angestellt und die Chondrite nicht nur nach ihrer Zusammensetzung, sondern sogar nach ihrer Struktur in einer der Natur vollkommen entsprechenden Weise k"unstlich nachgebildet. Die charakteristischen K"ugelchen von Olivin und Enstatit entstanden durch Schmelzung und Abk"uhlung von Magnesiasilikaten, also auf einem ganz andren Wege, wie die ganz analogen "`Organismen"' des Herrn Dr. Hahn! Auch Meunier stellte den Chondren ganz entsprechende Formen k"unstlich dar. Aus der Analogie der Chondren mit Hagelk"ornern schlie"st G"umbel, dass erstere "`durch Ansammlung Mineral bildender Stoffe in D"ampfen, unter gleichzeitiger drehender Bewegung"' entstanden sind; die ungew"ohnliche Entstehungsweise erkl"art hinl"anglich die ungew"ohnlichen Eigenschaften.

Die Chondren zeigen alle so viel "ubereinstimmung in ihrem Auftreten und ihrem Habitus, dass wir berechtigt sind, f"ur alle dieselbe Entstehungsart anzunehmen. Erweisen sich einzelne Chondren als Zoomorphosen, so muss es sich auch von allen "ubrigen nachweisen lassen, dass sie mineralisierte Tierreste sind; gelingt es, umgekehrt, nachzuweisen, dass die Struktur einzelner Chondren eine rein anorganische sei, dann muss dies f"ur die Chondren "uberhaupt gelten. Dieser Ansicht gem"a"s glaubte ich mit R"ucksicht auf die von mir als anorganisch erkannte Struktur der im Meteorstein von Tieschitz in M"ahren (15. Juli 1878) vorkommenden Chondren gegen die "`Unzweifelhaftigkeit"' der Hahnschen "`Organismen"', die bald als Pflanzen, bald als Schw"amme, dann wieder als Korallen und Crinoiden erscheinen, einige Zweifel aussprechen zu d"urfen. Wenn Herr Dr. Hahn meint, dass ich vorher seine Pr"aparate h"atte studieren sollen, dann gesteht er ja selbst zu, dass sein mit einem bedeutenden Kostenaufw"ande publiziertes Werk nicht geeignet sei, die Leser zu "uberzeugen; es w"are also gewiss zweckm"a"siger gewesen, das viele Geld zu sparen und die "`unzweifelhaftesten"' Organismen in der Welt "`zur gef"alligen Ansicht"' herum zu senden. Auf diese Art h"atte Herr Hahn f"ur seine "`Entdeckung"' und deren weltenumst"urzende Konsequenzen die beste Propaganda machen k"onnen!

In der Ausbildung der Chondrenstruktur gibt es wohl graduelle Verschiedenheiten, aber auch nur solche; der Typus bleibt immer derselbe. Weist ja Herr Dr. Hahn selbst auf den einheitlichen Typus seiner Organismen hin, ohne zu wissen, dass er damit einen gewichtigen Einwurf gegen seine eigenen Deutungen ausspricht. Die "ubergangsreihe zwischen den einzelnen Strukturformen kann, wie ich an diesem Orte (S. 396) dargelegt habe, unm"oglich als eine genetische (im Sinne der organischen Naturwissenschaften) betrachtet werden.

Auf die Exzentrizit"at der Struktur legt Herr Dr. Hahn besonders Gewicht. Was hat es nun aber f"ur ein Bewandtnis mit solchen Chondren, bei welchen die angeblichen "`Polypenr"ohren"' nicht exzentrisch, sondern gegen einen innerhalb der Chondren-Peripherie gelegenen Punkt zusammenlaufen? Solche Chondren sind allerdings selten, aber sie kommen doch vor; ich beobachtete ein solches Exemplar in einem D"unnschliff des Meteoriten von Tieschitz, und auch G"umbel und Tschermak konstatierten solches Vorkommen. Besonders interessant ist ein K"ugelchen, welches der letztgenannte Gelehrte im Meteorstein von Orvinio beobachtete und in welchem die quergegliederten S"aulchen ("`Crinoidenarme"') aus zwei innerhalb des Umrisses gelegenen Punkten gegen einander ausstrahlen! G"umbel sagt "uber die Struktur der Chondren: "`Zuweilen sieht es aus, als ob in einem K"ugelchen gleichsam mehrere, nach verschiedenen Richtungen hin strahlende Systeme vorhanden w"aren oder als ob gleichsam der Ausstrahlungspunkt sich w"ahrend der Bildung ge"andert h"atte, wodurch bei Durchschnitten nach gewissen Richtungen eine scheinbar wirre, st"angliche Struktur zum Vorschein kommt."' Eine solche wirre Lage der S"aulchen tritt nicht selten bei den Chondren des Tieschitzer Meteoriten auf, Tschermak beobachtete sie auch an den Chondren des Meteorsteines von Grosnaja (Kaukasus). Auch die Abbildungen, die Herr Dr. Hahn seinem Werke beigegeben hat, zeigen zum Teile eine verworrene Lage der S"aulchen.

Chondren dieser Art lassen sich wohl kaum als Organismen deuten; ist aber ihre Struktur als eine anorganische erkannt, dann ist es unstatthaft, die Struktur der gew"ohnlichen exzentrisch-faserigen Chondren f"ur eine organische zu erkl"aren.

Was das Vorkommen von Kan"alen, Durchbohrungen und Querscheidew"anden anbelangt, so werden diese bei den "`Organismen"' der Meteorite wahrscheinlich als ebenso anorganische Gebilde erkannt werden, wie die Kan"ale, das "`intermediate skeleton"' und die Kammerung des \emph{Eozoon canadense}.

Die in Kalkspatkristallen vorkommenden, geradlinigen Kan"ale sind allen Mineralogen bekannt, G. Rose hat sie ausf"uhrlich beschrieben. Sie stehen in Beziehung zu dem molekularen Bau des Kristalls. Bedeutungsvoller mit R"ucksicht auf die Kan"ale der Chondrenfasern d"urften jene haarfeinen, geradlinigen Kan"ale sein, welche zuerst G. Rose im Olivin des Pallaseisens erkannte und die sp"ater (1870) von N. v. Kokscharow beschrieben wurden. Die betreffenden Olivine waren Fl"achenreiche Kristalle!!

In dieselbe Kategorie d"urften eigent"umliche, in einem K"ugelchen des Meteorsteines von Lancé, von R. v. Drasche beobachtete Gebilde geh"oren. Das K"ugelchen zeigte mehrere, aus einem exzentrisch liegenden Punkte unter Winkeln von etwa 45° gegen die R"ander ausstrahlende Leistchen, an welche wieder andre, k"urzere, unter gleichem Winkel und in gr"o"serer Anzahl befestigt erschienen. Die letzteren Leistchen erschienen bei starker Vergr"o"serung hohl und teilweise mit einer dunkelgr"unen, flockigen Substanz erf"ullt. Diese kanalisieren Leistchen kann man mit R"ucksicht auf ihre geometrische Anordnung wohl kaum f"ur Korallenr"ohren oder Crinoiden halten. Vielleicht macht Herr Dr. Hahn ein neues Genus daraus, welches den "ubergang der Tiere in die – Mineralien vermittelt.

Im Querschnitte machen die Kan"ale nat"urlich den Eindruck von runden "offnungen; auch Glas- oder Gaseinschl"usse k"onnen so angeordnet sein, dass man sie leicht f"ur Perforationen halten kann. Ich beobachtete solche Einschl"usse in einem Kristall des Tieschitzer Meteoriten; da mir hierbei die Mineralsubstanz selbst ganz gleichg"ultig sein konnte, sprach ich mich "uber die mineralogische Natur dieses Kristalls in meiner Kritik des Hahnschen Werkes nicht n"aher aus. Sonderbarerweise hat das Fragezeichen, welches ich dem Worte "`Feldspat"' beif"ugte, den Zorn der Herren Hahn und Weinland so erregt, als ob hier einzig und allein die Substanz in Betracht zu ziehen w"are. Die Bestimmung der die Meteoriten zusammensetzenden Mineralien ist bekanntlich durchaus nicht so einfach, und selbst Koryph"aen auf diesem Gebiete bedienen sich, wie man sich aus der bez"uglichen Literatur "uberzeugen kann, weit h"aufiger des Wortes "`scheint"', als des Wortes "`ist"'. Niemand wird darin eine Unwissenheit, sondern eher nur eine Bescheidenheit erblicken, die gegen die grenzenlose Anma"sung, welche in dem von Herrn Dr. Hahn so oft gebrauchten Worte "`unzweifelhaft"' liegt, gewiss sehr angenehm absticht.

Die Quergliederung der Chondrenfasern ist oft ganz unregelm"a"sig, bei manchen Chondren nur stellenweise, bei manchen gar nicht ausgebildet. In den von mir beobachteten Chondren wird die Gliederung durch einfache Querkl"ufte bewirkt, die, wenn sie von fremder Substanz erf"ullt sind, leicht als Querw"ande erscheinen k"onnen. Im Meteorstein von Lancé find die Spaltungsdurchg"ange des Bronzits sehr oft von fremder Substanz erf"ullt; es entstehen dann nat"urlich scheinbar mit W"anden versehene R"ohren; ist die Zwischenlagerung fremder Substanz diskontinuierlich, so erscheinen die W"ande gleichsam durchbrochen. Manche Chondren zeigen eine wahrscheinlich aus Meteoreisen (G"umbel) bestehende "uberrindung, andre eine hellere, gegen den zentralen Teil sich abhebende Au"senzone; Chondren der letzteren Art kommen im Meteorstein von Grosnja und in dem von Tieschitz, h"ochst wahrscheinlich auch in andern Chondriten vor. Manchmal erscheinen die Chondren von au"sen her eingedr"uckt, in einer Weise, welche einen urspr"unglich plastischen Zustand der Chondren vermuten l"asst. Fast alle Bestandteile des Tieschitzer Meteoriten, n"amlich Olivin, Bronzit, Enstatit und Augit enthalten h"aufige Glaseinschl"usse; dieselben sind meist gestreckt, und erscheinen dann kanalartig; manchmal sind die m"aandrisch oder netzartig verteilt. Dieses Vorkommen der Glaseinschl"usse deutet auf sehr hohe Bildungstemperaturen der chondritischen Mineralien.

Wie die "`kreisrunden, elliptisch geformten Fl"achen mit einer Wand"' aussehen, von welchen Herr Dr. Hahn (\emph{Ausland}, Nr 26) spricht, kann ich mir trotz aller M"uhe nicht ganz deutlich vorstellen; wenn ich mich aber auch "uber dieselben nicht aussprechen kann, so glaube ich doch nachgewiesen zu haben, dass die Struktur vieler Chondren eine anorganische ist; nun sind aber "`alle die 100 Strukturformen"', welche die Chondren zeigen, durch zahllose "uberg"ange miteinander verkn"upft, wie die famose, hyperdarwinianische "`Entwicklungsreihe"' beweist, welche Herr Dr. Hahn mit mehr Verwegenheit als "uberlegung zwischen Schw"ammen, Korallen und Crinoiden aufgestellt hat.

Dass die "`100 Strukturformen"' sich auf einen einzigen Typus zur"uckf"uhren lassen, gesteht Herr Dr. Hahn selbst zu, und beantwortet also selbst die an mich (\emph{Ausland} Nr 26, S 504) gestellte Frage. Daf"ur blieb er mir bis zu diesem Augenblicke die Antwort schuldig auf meine Frage: "`Warum leugnet Herr Dr. Hahn die anorganische Natur des \emph{Eozoon canadense}, nachdem dieses Gebilde allen an die organische Natur der Chondren gekn"upften Bedingungen entspricht?"'

Das Meteoreisen erkl"art Herr Dr. Hahn f"ur "`Pflanzenfilz"', die Widmanst"attenschen Figuren f"ur Pflanzenzellen. Ich erlaube mir, Herrn Dr. Hahn darauf aufmerksam zu machen, dass man, wie Daubrée gezeigt hat, in nicht meteorischem Eisen eine den Widmannst"attenschen Figuren v"ollig analoge Struktur hervorbringen kann. Schon S"ommering erkannte (1816), dass die Linien der Widmannst"attenschen Figuren sich unter Winkeln von 60°, 90° und 120° schneiden, welche Winkel dem Oktaeder und W"urfel entsprechen. Am Braunauer Eisen lassen sich durch "atzen die W"urfelfl"achen leicht auffinden; andres Eisen zeigt deutlich oktaedrische und selbst tetraedrische Bl"atterdurchg"ange. Wenn Herr Dr. Hahn die Beobachtungen von Karsten "uber die Aufnahme von Eisen durch Pflanzenzellen f"ur seine Ansicht verwerten will, dann muss er auch trachten, die Art und Weise, wie die Reduktion des von den Zellen nicht im metallischen Zustande aufgenommenen Eisens erfolgen konnte, darzulegen. Da wird es jedoch notwendig sein, vorher ein wenig Chemie zu studieren!

Staunenswert ist es, dass Herr Dr. Hahn das Vorkommen von Kohle und Kohlenstoffverbindungen in manchen Meteoriten f"ur seine Ansichten nicht verwertet hat. Indem ich Herrn Hahn auf diesen Umstand aufmerksam mache, bin ich zugleich so grausam, ihm mitzuteilen, dass zwei M"anner, denen man immerhin gestatten darf, in dieser Angelegenheit ein W"ortchen mit drein zu reden, n"amlich Daubrée und Bischof, "uber den Kohlenstoffgehalt der Meteorite dem Herrn Dr. Hahn keinesfalls konvenierende Ansichten ausgesprochen haben.

Es wird mich gewiss sehr freuen, wenn es einmal gelingt, Organismen in Meteoriten mit Sicherheit nachzuweisen und dadurch unsern kosmogenetischen Theorien eine reale St"utze zu verleihen. Ich bin kein Ungl"aubiger von der Sorte eines J. de Luc, welcher erkl"arte, der Ansicht Chladnis "uber den kosmischen Ursprung der Meteoriten selbst dann nicht beipflichten zu wollen, wenn ihm "`ein Stein vom Himmel zu den F"u"sen niederfiele"'. Die bisherigen Ausf"uhrungen des Herrn Dr. Hahn und meine eigenen Beobachtungen haben mich von der organischen Natur der Chondren noch nicht "uberzeugt.

Es hei"st, Herr Dr. Hahn w"are kein "`Fachmann"'; dieser Umstand entschuldigt keineswegs die in seinen Publikationen enthaltenen fachlichen Missgriffe und Folgerungen. Wie kann ein Laie, d. h. Nicht-Fachmann, sich unterfangen, mit apodiktischer Gewi"sheit und einer alle Einw"urfe von vornherein abweisenden "`Unzweifelhaftigkeit"' Behauptungen aufstellen, die mit den Errungenschaften der Wissenschaft im Widerspruch stehen? Wie darf man es wagen, eine in die Gebiete der Pal"aontologie, Geogenie, Mineralogie und Chemie gleich tief eingreifende Frage zu er"ortern, ohne mit den genannten Disziplinen entsprechend vertraut zu sein?

Mit Ungeduld sehe ich den Beweisen entgegen, welche Herr Dr. Weinland, der selbst zugesteht, den Deutungen seines Freundes Hahn "`durchaus nicht "uberall folgen zu k"onnen"', f"ur die organische Natur der Chondren beibringen wird. Hoffentlich wird er als Fachmann mit weniger Anma"sung und mehr positiven Kenntnissen zu Werke gehen!

Br"unn, im Juli 1881.
\end{document}
