\documentclass[a4paper, 12pt, oneside]{article}
\usepackage[OT1]{fontenc}
\usepackage{booktabs}
\setlength{\emergencystretch}{15pt}
\usepackage{fancyhdr}
\usepackage{microtype}
\begin{document}
\begin{titlepage} % Suppresses headers and footers on the title page
	\centering % Centre everything on the title page
	\scshape % Use small caps for all text on the title page

	%------------------------------------------------
	%	Title
	%------------------------------------------------
	
	\rule{\textwidth}{1.6pt}\vspace*{-\baselineskip}\vspace*{2pt} % Thick horizontal rule
	\rule{\textwidth}{0.4pt} % Thin horizontal rule
	
	\vspace{1\baselineskip} % Whitespace above the title
	
	{\LARGE Yet Again the} % Title
	
    {\LARGE ``Organisms of the Meteorite''} % Title
	
	\vspace{0.5\baselineskip} % Whitespace below the title
	
	\rule{\textwidth}{0.4pt}\vspace*{-\baselineskip}\vspace{3.2pt} % Thin horizontal rule
	\rule{\textwidth}{1.6pt} % Thick horizontal rule
	
	\vspace{1\baselineskip} % Whitespace after the title block
	
	%------------------------------------------------
	%	Subtitle
	%------------------------------------------------
	
	{Anton Rzehak} % Subtitle or further description
	
	\vspace*{1\baselineskip} % Whitespace under the subtitle
	
    {\small Das Ausland, No. 37, Article 4} % Subtitle or further description
    
	%------------------------------------------------
	%	Editor(s)
	%------------------------------------------------
	
	\vspace{1\baselineskip}

	{\small\scshape Stuttgart --- September 12, 1881 \\ }
			
    \vspace*{\fill}

    Internet Archive Online Edition  % Publication year
	
	{\small Attribution NonCommercial ShareAlike 4.0 International } % Publisher
\end{titlepage}
\setlength{\parskip}{1mm plus1mm minus1mm}
\clearpage
\paragraph{}
I am at a loss as to whether or not it is an advantage for science if the representatives of it display a certain indifference towards literary works that can easily be misinterpreted. "\emph{Qui tacet, consentire videtur}"; according to this tenet the vast public is swayed and empowers the most audacious hypothesis, and unless no objections are raised from an authoritative side it will turn into a dogma. While the academic is content to allow a degree of likelihood, the common person with justice and by right may request enquiries into genuine truth; the entire complicated apparatus of scientific research and activity, the many faceted merging and interaction of the different specialities is to him completely foreign. Such achievements, which grab the general curiosity, will become before long well-known and consequently to all the educated "of this day and age" it will become invaluable scientific research preserved in the pantry "for household use." The contact between scholars and the public is in most cases mediated only through daily journalism; the mediator is in general unable to apply the standard of academic critique itself, he nevertheless must still aspire to factually satisfy the needs of the public. And so every now and then they pick from the tree of science a fruit and offer it up for enjoyment, without even considering whether or not this fruit is ripe and edible. It is in this way that different views happen to become disseminated, about which scholars have by no means become agreed, and accepted as \emph{bona fide} facts by the public. And so it was with the "organisms of the meteorite"; the "discovery" of Dr. Hahn has been talked about in numerous publications without any critique and seems ready to become quite popular, even before it is confirmed or refuted by a qualified side. So far \emph{pro als contra} have been advanced by only a few voices, even though the issue is undeniably of profound significance for the entire monistic weltanschauung. The possibility of organized structures existing in the meteorites is by no means excluded from the start and this claim should be asserted not only with likelihood but with certainty, arrived at by a professional near to the matter whose duty is to undergo a neutral critique without bias. How has it come to be that in general one shies from openly expressing their judgement on such an interesting question? One is involuntarily reminded of the anxiousness with which scholars at the beginning of this century sought notions to evade Chladni's assertion about the origin of the meteorites. People alleged at the time that "Chladni had merely thrown out a paradoxical notion, and with all imaginable pretexts they rigged up a way around and, once the physicists seriously followed suit, made fun of it." Perhaps there are similar anxieties regarding the Hahnian "discovery"; however, do you believe that the spawn of amateurs can be rendered harmless by just completely ignoring it?

Years ago Dr. Jenzsch, a counsellor of mines and the forerunner of Dr. Hahn, believed that he had discovered the fossil remains of organisms in melaphyritic and porphyritic rocks; although he did not whimsically arrive at corals and crinoids, he mentioned obtaining perfectly well-preserved algae, infusoria, and rotifers. J. G. Bornemann, at the Nature Research Assembly of Dresden (1868), reviewed and determined "that amongst all the alleged animal and plant remains not the slightest could be found, the structures should have been interpreted in a natural way as inorganic apparitions and as having arisen in a clear physical manner." Can one blame Bornemann, lest he hold it beneath his dignity to verify the views of Jenzsch? Certainly Not!

Dr. Hahn is no longer isolated in his view; he has found in Dr. Weinland a defender, who has further convinced a German paleontologist, whose name regrettably was kept secret, of the zoomorphic nature of the chondrules. It is therefore advisable under these circumstances to engage in the impartial examination of the matter and with this I myself call for anyone who can to take the opportunity to scrutinize thin sections of chondrites. Dr. Hahn need not fret at this request; if his views are right, in spite of all attacks, then they will finally become accepted as so.

The essential question to debate is simply: "Is the structure of the chondrules purely mineralogical or not?"

Most meteorite experts will no doubt answer in the affirmative without further thought; one must however strive to explain and prove in "black and white," with as many arguments as possible demonstrating the inorganic structure of the chondrules, so as not to be accused of Dr. Hahn's "superficiality" and "dishonesty."

The idiosyncrasies of the chondrites have already been highlighted by [Gustav] Rose, and probably everyone who has had the opportunity to scrutinize them will reach the same conclusion, that their method of formation is unlike any known method of formation of a terrestrial rock. The analogy of the latter with the chondrites, despite some similarities, is but an imperfect one. Gümbel explains the chondrites as clastic rock and Tschermak finds in their peculiar structure certain links to terrestrial tuffs; having said this, he is reminded of the trituration of rigid masses and excludes the action of water during the formation of the chondrite from the outset.

In the opinion of Dr. Hahn, the chondrites would have to be purely clastic rock, which became sedimentary deposition in very calm water, since "nowhere are there tumbled forms or flakes." Having said this Dr. Hahn then says, "that the rock of the chondrites is not quite similar to our sedimentary rock, a slurry in which the animals became embedded." The "entire mass" is said to be comprised of organisms; if this is the case, it remains quite puzzling as to what the crinoids, corals, and sponges, whose growth spot Dr. Hahn has quite clearly noted, were actually attached??

By no means do the chondrites demonstrate a significant agreement with the clastic rocks of the Earth's crust. According to Gümbel's point of view the meteorites are supposed to emerge from "a kind of primal slagging process of the celestial bodies." As is generally known, Daubrée contrived a very interesting synthetic experiment on the method of formation of the meteorites and replicated the chondrites not only in their composition but also in their structure in an artificial manner fitting with nature. The characteristic balls of olivine and enstatite formed through a melting and cooling of magnesium silicates, hence an entirely different way from all the entirely analogous "organisms" of Dr. Hahn! Meunier also made artificial forms analogous to the chondrules. Based on the analogy of the chondrules with hailstones, Gümbel reasons that the former were formed "thru the agglomeration of mineral forming substances in vapor together with a simultaneous rotating movement"; the unusual manner of formation sufficiently explains the unusual features.

The chondrules display so much conformity in their occurrence and habitus that we are able to assume the same kind of formation for all of them. If individual chondrules are proven as zoomorphic, then all the remaining ones must also be granted as mineralized animal remains; conversely, if it is successfully proven that the structure of individual chondrules is purely inorganic, then this must hold true for all the chondrules in general. In accordance with this notion, I am inclined to regard them as inorganic structures, based on the chondrules in the Tieschitz meteorite from Moravia (July 15, 1878), contrary to the "undoubtable" organisms of Dr. Hahn, manifesting sometimes as plants, sometimes as sponges, then again as corals and crinoids, one may be permitted to express a few doubts. If Dr. Hahn reckons that I should have studied his slides beforehand, then he himself admits that his work, published with great expense, itself is not suitable to convince the readers; thus it certainly would have been more expedient to save all the money and send around the "unquestionable" organisms to the public "for pleasing opinions." In this way Dr. Hahn could have gained favor for his "discovery" and its world overturning consequences which make fine propaganda!

The formation of the chondritic structure probably allows slight disparities, but only such; the type always remains the same. Indeed, Dr. Hahn himself pointed out the consistent type of his organisms, without knowing that he was thereby expressing a serious objection against his own interpretations. The sequence of transitions among the individual structural forms, as I have shown here (p. 396), cannot possibly be regarded as a genetic one (in the sense of the organic natural sciences).

Dr. Hahn puts extra significance on the eccentricity of the structure. But what is the reason for such chondrules, in which the so-called "tube polyps" do not intersect eccentrically, but rather come together at a spot located within the chondrule-periphery? Such chondrules are indeed rare, but they happen nevertheless; I observed one such specimen in a thin section of the Tieschitz meteorite, and even Gümbel and Tschermak noted such occurrences. Especially interesting is one globule, observed in the Orvinio meteorite by the latter scholar, in which the transversely structured small columns ("crinoid arms") radiate towards each other from two points located within the outline! Gümbel says about the structure of the chondrules: "Sometimes it seems, so to speak, as if a number of systems radiating towards distinct directions exist in one globule or as if, so to speak, the radiance point itself was altered during its formation, so that by intersecting in certain directions a seemingly tangled columnar structure emerges." Such a tangled state of the small columns is not unusual in the chondrules of the Tieschitz meteorite, Tschermak even observed it in the chondrules of the Grosnaja meteorite (Caucasus) [Mekenskaya, Chechnya, Russia]. Even the photographs added by Dr. Hahn to his work display, to some extent, an entangled state of the small columns.

Chondrules of this type hardly allow themselves to be interpreted as organisms; however, once their structure is recognized as inorganic, it then becomes inadmissible to interpret the ordinary filamentous eccentric chondrules as organic.

Concerning the existence of channels, tubular penetration, and transverse-partition walls, these "organisms" of the meteorite will likely turn out to be recognized as inorganic formations just like the channels of the "intermediate skeleton" and chambering of the \emph{Eozoon canadense}.

The rectilinear channels existing in calcite crystals are familiar to all mineralogists, G. Rose has described them extensively. They are related to the molecular construction of the crystals. More significant, related to the channels of the chondrule fibers, are those hair-thin rectilinear channels that G. Rose first identified in the olivine of the pallasites and which were later (1870) described by N. v. Kokscharow. The olivines in question were richly-faceted crystals!!

One may consider a specific type which belongs to the same category, a form observed by R. v. Drasche in globules of the Lancé meteorite. The globules displayed a number of battens radiating from an eccentric recumbent point at angles of approximately 45$^\circ$ to the edges, to which again other, shorter ones with similar angles and in larger numbers appeared attached. The previous battens appear largely hollow under magnification and partially suffused with a dark green, flocculent substance. These channeled battens can hardly be considered as coral tubes or crinoids, given their geometrical arrangement. Perhaps the authority of Dr. Hahn made a novel genus out of it, which mediates the transition of the animals from the – minerals.

In cross sections the channels naturally give the impression of round openings; as even glass or gas inclusions are capable of being arranged in such a way that they could easily be considered by someone as perforations. I observed such inclusions in a crystal of the Tieschitz meteorite; because I was indifferent to the mineral substance itself, I did not talk in great detail about the mineralogical nature of these crystals in my critique of Hahn's work. Strangely enough, the question mark which I added to the word "feldspar" aroused the anger of Messrs. Hahn and Weinland, as if the only thing in consideration here was the substance alone. The identification of the minerals composing the meteorites is, as is well known, not so straightforward, and even luminaries in this field employ, as one can be convinced from the relevant literature, the word "seems" far more frequently than the word "is." No one will see this as ignorance, if anything simple humility, as opposed to the unbounded arrogance so often used by Dr. Hahn with words like "undoubted," certainly a very pleasant switch.

The cross structure of the fibrous chondrules is often quite irregular, displayed by many chondrules only in places, in some not at all. In the chondrules, which I have observed, the structure is produced by ordinary cross fissures, which, when they are suffused with foreign substances, can come across as transverse partitions. In the Lancé meteorite the cleavage openings of bronzite are frequently pervaded by foreign substances; these could naturally be mistaken as the illusive tubes with septa; if the deposit of the foreign substance is discontinuous such septa appear, as one might say, breached. Many chondrules display an outer layer presumably consisting of meteoritic iron (Gümbel), other ones a brighter outer zone disappearing in the center portion, chondrules of this latter kind occur in the meteorite of Grosnja and in that of Tieschitz; most likely in other chondrites as well. At times the chondrules appear impressed from the outside, in a way that allows one to suppose that the chondrules were originally in a plastic state. Almost all the constituents of the Tieschitz meteorite, namely olivine, bronzite, enstatite, and augite contain a lot of glass inclusions; these are usually elongated and thus seem channel-like; sometimes they meander or are arranged like in a net. This incidence of glass inclusions indicates very high formation temperatures for the chondritic minerals.

What these "circular, elliptically shaped areas with a wall" look like, as mentioned by Dr. Hahn (\emph{Das Ausland}, No. 26), despite much hassle myself, I am entirely unable to distinctly envisage; even though I am unable to describe these, I nevertheless think that I have established that many of the chondrules have an inorganic structure; but then how could "all the hundreds of structural forms" that the chondrules display be related collectively through some countless descent, as the famous hyper-Darwinian "sequence of development" demonstrates and which Dr. Hahn has established more with audacity than with consideration, between the sponges, corals, and crinoids.

That the "100 structural forms" can be traced back to a single type is suggested by Dr. Hahn himself and hence answers the question he placed (\emph{Das Ausland} No. 26, pg. 504) to me. For he has up till now continued to ignore my question: "Why does Dr. Hahn deny the organic nature of the \emph{Eozoon canadense}, since this formation fulfills all the conditions attached to the organic nature of the chondrules?"

Dr. Hahn declares the meteoritic iron as a "fine web of plants," the Widmanstätten patterns as plant cells. I allow myself to draw Dr. Hahn's attention that someone, namely Daubrée, has demonstrated that in non-meteoritic iron a completely analogous structure to that of the Widmanstätten patterns can be generated. Sömmering realized as early as 1816 that the lines of the Widmanstätten patterns intersect themselves at angles of 60$^{\circ}$, 90$^{\circ}$ and 120$^{\circ}$, angles which correspond to that of the octahedron and cube. Planes of a cube in the Braunau iron can be easily detected through etching; other irons clearly show octahedral and even tetrahedral sheet transits. If Dr. Hahn wishes to utilize the observations of Karsten about the assimilation of iron by plant cells to support his case, then he must also seek to try to elucidate the type and manner by which the reduction of iron not in a metallic state could be made to occur in the cells. Having said this, it will be necessary for him to study a little chemistry beforehand!

It is astonishing that Dr. Hahn did not exploit the existence of coal and carbon compounds in some of the meteorites for his slanting views. While making Dr. Hahn aware of these facts, I am at the same time sad to inform him that two men, would could be allowed to speak a few words on this matter, namely Daubrée and Bischof, about the carbon content of the meteorites, by no means expressed views in agreement with those of Dr. Hahn.

It would certainly make me very happy if one day it turns out that organisms in the meteorites can be proven with reliability, therefore imparting real support for our cosmogenetic theories. I am not a doubter of J. de Luc's sort, who proclaimed that he would never accede to Chladni's view on the cosmic origin of the meteorites, even if "a stone fell down from the sky to his feet." The statements of Dr. Hahn up till now, along with my own observations, have not yet convinced me of the organic nature of the chondrules.

It was mentioned that Dr. Hahn is not an "expert"; this fact in no way excuses the technical blunders and conclusions contained in his publication. How can a layman, i.e. a non-expert, himself venturing from the start, establish with apodictic sureness and all number of throw-ins and "unquestionable" claims that reject the achievements of science that stand in contradiction? How can one attempt to discuss an issue that profoundly impinges upon the fields of paleontology, geology, mineralogy, and chemistry, and not be familiar with the relevant disciplines?

I eagerly await the counterproof that Dr. Weinland, who himself concedes that he is "by no means able to follow everywhere" his friend Hahn's explanations go, will produce in favor of the organic nature of the chondrules. Hopefully as an expert he will go to work with less hubris and more affirmative knowledge!

Brünn, July 1881.
\end{document}
