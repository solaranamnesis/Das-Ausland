\documentclass[a4paper, 12pt, oneside]{article}
\usepackage[OT1]{fontenc}
\usepackage{booktabs}
\setlength{\emergencystretch}{15pt}
\usepackage{fancyhdr}
\usepackage{microtype}
\begin{document}
\begin{titlepage} % Suppresses headers and footers on the title page
	\centering % Centre everything on the title page
	\scshape % Use small caps for all text on the title page

	%------------------------------------------------
	%	Title
	%------------------------------------------------
	
	\rule{\textwidth}{1.6pt}\vspace*{-\baselineskip}\vspace*{2pt} % Thick horizontal rule
	\rule{\textwidth}{0.4pt} % Thin horizontal rule
	
	\vspace{1\baselineskip} % Whitespace above the title
	
	{\LARGE About the\\}
	\smallskip
	{\LARGE ``Organisms of the Meteorite''}
	
	\vspace{1\baselineskip} % Whitespace above the title

	\rule{\textwidth}{0.4pt}\vspace*{-\baselineskip}\vspace{3.2pt} % Thin horizontal rule
	\rule{\textwidth}{1.6pt} % Thick horizontal rule
	
	\vspace{1\baselineskip} % Whitespace after the title block
	
	%------------------------------------------------
	%	Subtitle
	%------------------------------------------------
	
	{Anton Rzehak in Brünn} % Subtitle or further description
	
	\vspace*{1\baselineskip} % Whitespace under the subtitle
	
    {\small Das Ausland, No. 20, Article 5} % Subtitle or further description
    
	%------------------------------------------------
	%	Editor(s)
	%------------------------------------------------
	
	\vspace{1\baselineskip}

	{\small\scshape Stuttgart --- May 16, 1881 \\ }
			
    \vspace*{\fill}

    Internet Archive Online Edition  % Publication year
	
	{\small Attribution NonCommercial ShareAlike 4.0 International } % Publisher
\end{titlepage}
\setlength{\parskip}{1mm plus1mm minus1mm}
\clearpage
\paragraph{}
When Dr. Otto Hahn's work \emph{The Meteorite (Chondrite) and its Organisms} came into my hands last year I was well aware of the importance that the detection of unquestionable organisms in meteorites would have for cosmology. After reading the above work, however, I had to confess to myself that the proof had not yet been provided with the desired certainty; I believe I aroused the same opinion in my auditorium when, at the March meeting of the Proceedings of the Natural History Society of Brünn, I spoke about Dr. Hahn's work.

I did not originally intend to announce in this way the view that I had formed about the ``organisms'' of the meteorites; I thought to myself that professional circles would, regardless, form their independent judgement and lay people would rarely get their hands on Hahn's book because of its high price, due to its furnishings. I am prompted by the article published in No. 16 of this journal by Dr. David F. Weinland under the title ``Corals in the Meteorites''.

The only criticism of Hahn's work that has come to my attention thus far is the one by the French Academy at the meeting of January 3, 1881. A French weekly (\emph{L'Illustration}) has communicated this criticism to its readers under the title ``A German Savant's Error''. Dumas, who had presented and discussed Hahn's book, first pointed out that according to Stanislas Meunier quite similar forms to that which Hahn considers to be organisms can be obtained through artificial means. Mr. Dumas seems to have succeeded in convincing the Academy of the incorrectness of Hahn's view because \emph{L'Illustration} speaks of a ``success of unanimous laughter''.

I mention here that I had the opportunity and still have the opportunity at every moment to examine several splendid specimens of organisms (3) in thin sections of the Tieschitz meteorite of Moravia (July 15, 1878), so that I am not accused of incompletely representing the ``too little'' photographic figures of Hahn's work.

Dr. Otto Hahn describes the chondrites as a ``felt of animals, a fabric whose meshes were all living beings''; Dr. Weinland recognizes in the inclusions in question, which can be referred to as ``chondrules'' with Gümbel, likewise ``undoubted animal remains''. In order to give all those who have not read Hahn's work a small idea of the ambiguity of these ``animal remains'' right from the outset, I note here that most of the ``animals'' were thought to be plants not long ago by Dr. Hahn!

On page twenty of his work, Dr. Hahn establishes the conditions in whose fulfillment shows, in his opinion, the proof of the organic nature of the chondrules. These conditions are:
\begin{enumerate}
    \item A closed form.
    \item A recurring form.
    \item Recurrence of form in stages of development.
    \item Structure (cells or vessels).
    \item Similarity with known forms.
\end{enumerate}
\paragraph{}
As far as the ``closed'' form is concerned, the word ``closed'' is supposed to indicate a specific outline consistent with the structure. For the ``organisms'' of the Tieschitz meteorite I must deny a closed form in this sense.

The ``recurring'' of the same form cannot provide an argument in assessing the organic or inorganic nature of the chondrules. Many microscopic mineral inclusions show ``closed'' and ``recurring'' forms without supposing the odds and ends of organisms in them.

Regarding the ``recurrence of form in stages of development'', I strongly say that there are no ``stages of development'' in the sense that Dr. Hahn takes, they do not exist and cannot serve as proof. It cannot be denied that a transitional series can be created between the structureless and the more complex forms of the chondrules; however, the resulting developmental series cannot be called a phylogenetic one (in the sense of organic science), and if Dr. Hahn lets crinoids emerge from corals and sponges ``through multiplication of the channels'', then this is a process which is completely incompatible with what we know about the phylogeny and ontogeny of protozoa, coelenterata, and echinoderms. It is precisely the ``uniform'' type of meteoritic organisms, highlighted on page thirty-three of Hahn's work, and the fact that all the forms can be placed in a transitional series that seem to me to constitute important arguments against the organic nature of the chondrules. Which zoologist or paleontologist would see a uniform type in sponges, corals, and crinoids?

The ``structure'' of the chondrules, on the whole, reminds one of certain tube corals and, if one wants to be lenient, one could forgive a layman for the confusion with terrestrial \emph{Favosites}. Some chondrules show no structure; these are considered the most primitive and Dr. Hahn, as well as Dr. Weinland, takes them for sponges. If a structure with more or less radial columns is noticed, especially if there are also transverse partitions (which is not always the case), then there arises an ``undoubted'' tube coral. If a central longitudinal channel passes through the transversely dividing columns, the ``undoubted'' crinoid is good-to-go. The development is sometimes so rapid that a sponge directly turns into a crinoid. Such an advancement was made, for example, in the specimen depicted by Dr. Hahn in Table 30: Figure 5; it is an ``undoubted'' crinoid who, with all the pride of a parvenu, can look back to the dark days when he lived as a ``sponge'' in the collection of Dr. Hahn. Gümbel compared the structure of the chondrules, which I want to describe as ``favositoid'', with the structure of hailstones, a comparison that can be called apt in every respect. The excentricity of the radiation point of the fibers is probably the rule, but I found an inclusion in the Tieschitz meteorite in which the fibers meet within the sphere. I was also able to confirm several times the observation of Gümbel that in some pellets (chondrules) ``there are several radiating systems with different directions'' and thus a ``seemingly confused, channel structure'' comes to light. The favositoid structure of the chondrules is only one of the formations with the ``columnar'' structure, which also occurs in other inclusions of the chondrites; the latter I could observe in a feldspar (?), whose rectilinear outlines are quite clearly recognizable; the slats, respectively columns, are probably not radially arranged, but are particularly interesting because in the middle of several are found noticable round glass inclusions arranged in a longitudinal row. Such small inclusions seem to be thought of as perforations analogous to those found in the tube walls of the \emph{Favosites}. Sometimes the individual roundish droplets blur into an apparent channel passing through the center of the column. The supposed wall openings can also be found where no transverse partitions divide the ``coral tube''. The transverse partitions can be seen very often and, where they are developed, reveal themselves by the irregularity and indeterminacy of their appearance as simple transverse fissures, as I could observe them in macroscopic formations of the enstatite of Zdjar and in the tourmaline columns of Rozna in Moravia. It is impossible to consider the ``transverse partitions'' of the chondrules as real transverse walls formed by organic activity and analogous to the dissepiments of terrestrial corals. Gümbel, who is familiar with micropaleontological investigations, would certainly have recognized the organic structure of the ``fine tranversely segmenting fibers'', if one were dealing with such phenomenon at all.

As far as the similarity of the chondrules with known forms is concerned, at most it is an external one. Can an object, which if first declared to be a plant, then a sea sponge, and finally a crinoid resemble a ``known form''? I am confident that nobody, not even Proteus, could form a clear presentation.

It is clear from what has been said that the five conditions issued by Dr. Hahn do not at all imply proof of an animal nature of the chondrules. If (p. 33) the ``correspondence of similar forms'' is regarded as an ``important point of evidence'' for an organic nature, then with the same degree of probablity the augite crystals of a lava or the houses of a city should be regarded as organisms. How is it, by the way, that Dr. Hahn denies the organic nature of the \emph{Eozoon canadense}, even though it fulfills all the conditions he has issued? Dr. Hahn takes the most primitive forms of the chondrites, as already noted, for sponges and summarizes them under the name ``\emph{Urania}''; he finds great affinity between them and terrestrial forms and even recognizes the genus \emph{Astrospongia} (!). He can clearly distinguish the growth sites and mouth openings at the thin throats of his sponges. Dr. Hahn considers indistinct tangles of small crystal bands to be needle spicules of sponges; in the case of a possible ``advancement'' of such a needle sponge to a crinoid, the needles cannot of course remain as impossible needles but must become crinoid arms. Dr. Hahn's zoological escamotage, causing the blood in the enraged Darwinists' veins to solidify, which he has indeed accomplished, can be seen on page twenty-five of his work. In any case, this places the ``undoubted'' animal nature of the chondrules in quite a strange light.

As far as the ``corals'' are concerned, a comparison or even identification of them with terrestrial forms is not permissible; since most of the ``colonies'' are only fractions of a millimeter in diameter, the dimensions of the individual ``polyp tubes'' one finds are so small that there is no justification in assuming that these microscopic colonies were once inhabited by animals with a close relationship to terrestrial anthozoa. For this reason, Dr. Weinland raised the ``\emph{Favosites}'' of the chondrites to a new genus, which he calls ``\emph{Hahnia}''.

The differentiation between cup, tube, and star corals indicated to me that Dr. Hahn, apart from everything else, had gone too far.

According to Dr. Hahn, the crinoids are found to be ``from the simplest form with an articulated arm, to the developed crinoids with stem, crown, main and auxiliary arms''. Addressed as crinoids, e.g. Figures 1 and 2 of Table 25; but they do not look like it at all, for the alleged crinoid arms are everywhere the same width and quite simple, while, as is well known, they actually taper away from the crown and usually branch. The structure of the ``arms'' is so irregular and imperfect that, of all the known crinoids, no one is reminded of one. The ``kinking'' of the arms can only explained, according to Dr. Hahn's view, by crinoids; if this kinking is not there, Dr. Hahn declares the undoubted crinoids as an equally undoubted coral! After finding one of the above-mentioned, cross-sectioned enstatite crystals also kinked, must I also consider it as a ``crinoid arm''?

Some ``crinoids'' consist, according to Dr. Hahn, ``merely as any number of arms''; the stems and crowns seem to be missing from these crinoids, and Dr. Hahn therefore finds it completely justifiable to declare them as a ``special type''. Declaring them as ``special'' crinoids would be like claiming a fish consisting only of fins was special.

It may be of interest to many to learn that Dr. Hahn has observed in many of his crinoids not only the stem and crown, but also the ``mouth opening between the bulge'', and — hear and be amazed — even clearly observed muscle layers!!

If one compares the alleged organisms of the chondrites with terrestrial forms, one must presuppose similar conditions of existence; from this requirement one must consistently conclude that the chondrites are to be regarded as an analogue of terrestrial clastic rocks. Against this logically necessary result, Dr. Hahn decidedly pronounces a mode of formation for the chondrites which substantially alters our previous views on cosmology. However, if one goes by the premises set out by Dr. Hahn and draws conclusions in a strictly logical manner, one soon finds oneself in a chaos of contradictions which are absolutely impossible to solve.

From the chemist's point of view one could also make many objections to Hahn's work; however, I do not want to go into it any further and only mention that such views as developed by Dr. Hahn, e.g. on the origin of the mountains and volcanoes, cannot be forgiven even by a layman nowadays.

Brünn, April 25, 1881.
\end{document}
