\documentclass[a4paper, 12pt, oneside]{article}
\usepackage[utf8]{inputenc}
\usepackage[T1]{fontenc}
\usepackage[ngerman]{babel}
\usepackage{fbb} %Derived from Cardo, provides a Bembo-like font family in otf and pfb format plus LaTeX font support files
\usepackage{booktabs}
\setlength{\emergencystretch}{15pt}
\usepackage{fancyhdr}
\begin{document}
\begin{titlepage} % Suppresses headers and footers on the title page
	\centering % Centre everything on the title page
	\scshape % Use small caps for all text on the title page

	%------------------------------------------------
	%	Title
	%------------------------------------------------
	
	\rule{\textwidth}{1.6pt}\vspace*{-\baselineskip}\vspace*{2pt} % Thick horizontal rule
	\rule{\textwidth}{0.4pt} % Thin horizontal rule
	
	\vspace{1\baselineskip} % Whitespace above the title
	
	{\LARGE "Uber die\\}
	\smallskip
	{\LARGE "`Organismen der Meteorite"'}

	\vspace{1\baselineskip} % Whitespace above the title

	\rule{\textwidth}{0.4pt}\vspace*{-\baselineskip}\vspace{3.2pt} % Thin horizontal rule
	\rule{\textwidth}{1.6pt} % Thick horizontal rule
	
	\vspace{1\baselineskip} % Whitespace after the title block
	
	%------------------------------------------------
	%	Subtitle
	%------------------------------------------------
	
	{Anton Rzehak in Br"unn} % Subtitle or further description
	
	\vspace*{1\baselineskip} % Whitespace under the subtitle
	
    {\small Das Ausland, Nr. 20, Artikel 5} % Subtitle or further description
    
	%------------------------------------------------
	%	Editor(s)
	%------------------------------------------------
	
	\vspace{1\baselineskip}

	{\small\scshape Stuttgart --- May 16, 1881\\}
			
    \vspace*{\fill}

    Internet Archive Online Edition  % Publication year
	
	{\small Namensnennung Nicht-kommerziell Weitergabe unter gleichen Bedingungen 4.0 International} % Publisher
\end{titlepage}
\setlength{\parskip}{1mm plus1mm minus1mm}
\clearpage
\paragraph{}
Als mir in der Herber vorigen Jahres Dr. O. Hahns Werk "uber de \emph{Meteorite (Chondrite) und ihre Organismen} in die H"ande kam, da war ich mir der Wichtigkeit wohl bewusst, welche der Nachweis unzweifelhafter Organismen in Meteorsteinen f"ur die Kosmogonie haben m"usste. Nach Durchlesung des genannten Werkes musste ich mir jedoch gestehen, dass dieser Nachweis bisher noch keineswegs mit der w"unschenswerten Sicherheit erbracht worden ist; dieselbe Meinung glaube ich auch in meinem Auditorium erweckt zu haben, als ich in der M"arzsitzung des naturforschenden Vereines in Br"unn "uber Dr. Hahns Werk referierte.

Ich war urspr"unglich nicht gesonnen, die Ansicht, welche ich mir "uber die "`Organismen"' der Meteorite gebildet habe, auf diesem Wege bekannt zu geben; ich dachte mir, dass die die Fachkreise sich ohnedies ihr selbst"andiges Urteil bilden w"urden und Laien das Hahn'sche Buch seines durch die Ausstattung bedingten hohen Preises wegen nur selten in die H"ande bekommen. Zur Mitteilung vorstehender Zeilen sehe ich mich durch die in Nr. 16 dieser Zeitschrift von Dr. D. Fr. Weinland unter dem Titel "`Korallen in Meteorsteinen"' ver"offentlichte Notiz veranlasst.

Bisher war mir "uberhaupt nur eine Kritik des Hahn'schen Werkes bekannt geworden, und das ist jene, welche die franz"osische Akademie in der Sitzung vom 3. Januar 1881 ge"ubt hat. Eine franz"osische Wochenschrift (\emph{L'Illustration}) hat diese Kritik unter dem Titel "`Une erreur de savant allemand"' ihrem Leserkreis mitgeteilt. Dumas, welcher das Hahn'sche Buch vorgelegt und besprochen hatte, wies zuerst darauf hin, dass man nach Stan. Meunier ganz "ahnliche Formen, die Hahn f"ur Organismen h"alt, auf k"unstlichem Wege erhalten k"onne. Es scheint Herrn Dumas gelungen zu fein, die Akademie von der Unrichtigkeit der Hahn'schen Ansicht zu "uberzeugen, denn \emph{L'Illustration} spricht von einem "`succès de rires unanimes"'.

Ich erw"ahne gleich hier, dass ich Gelegenheit hatte und auch jetzt noch jeden Augenblick Gelegenheit habe, mehrere Prachtexemplare von Organismen (3) an D"unnschliffen des Meteoriten von Tieschitz in M"ahren (15. Juli 1878) zu untersuchen dass mich also der Vorwurf, nur nach den unvollkommenen, "`zu wenig"' darstellenden photographischen Tafeln des Hahn'schen Werkes geurteilt zu haben, nicht treffen kann.

Dr. O. Hahn bezeichnet die Chondrite als einen "`Filz von Tieren, ein Gewebe, dessen Maschen alle lebendige Wesen waren"'; Herr Dr. Weinland erkennt in den fraglichen Einschl"ussen, die man mit G"umbel kurz als "`Chondren"' bezeichnen kann, ebenfalls "`unzweifelhafte Tierreste"'. Um allen jenen, welche Hahns Werk nicht selbst gelesen haben, von der Unzweideutigkeit dieser "`Tierreste"' gleich von vorneherein einen kleinen Begriff zu geben, f"uhre ich hier an, dass der gr"oßte Teil der "`Tiere"' vor nicht langer Zeit von Herrn Dr. Hahn f"ur — Pflanzen gehalten wurde!

Auf S. 20 seines Werkes stellt Herr Dr. Hahn f"unf Bedingungen auf, an deren Erf"ullung sich nach seiner Ansicht der Beweis der organischen Natur der Chondren kn"upft. Diese Bedingungen sind:
\begin{enumerate}
    \item Eine geschlossene Form.
    \item Eine wiederkehrende Form.
    \item Wiederkehren der Form in Entwicklungsstufen.
    \item Struktur (Zellen oder Gef"aße).
    \item "Ahnlichkeit mit bekannten Formen.
\end{enumerate}
\paragraph{}
Was zun"achst die "`geschlossene"' Form anbelangt, so soll das Wort "`geschlossen"' wohl einen bestimmten, mit der Struktur im Einklang stehenden Umriss andeuten. F"ur die "`Organismen"' des Tieschitzer Meteoriten muss ich eine in diesem Sinne geschlossene Form in Abrede stellen.

Das "`Wiederkehren"' derselben Form kann doch unm"oglich ein Argument zur Beurteilung der organischen oder nicht organischen Natur der Chondren abgeben. Viele mikroskopische Mineraleinschl"usse zeigen "`geschlossene"' und "`wiederkehrende"' Formen, ohne dass man in ihnen Organismenreste vermutet.

Betreffs der "`Wiederkehr der Form in Entwicklungsstufen"' spreche ich mich ganz entschieden dahin aus, dass es "`Entwicklungsstufen"' in dem Sinne, wie sie Herr Dr. Hahn nimmt, nicht gibt und auch gar nicht geben kann. Es ist wohl nicht zu leugnen, dass man zwischen den einfachsten, ungegliederten und den komplizierteren Formen der Chondren eine "Ubergangsreihe herstellen kann; die so resultierende Entwicklungsreihe kann jedoch keineswegs eine phylogenetische (im Sinne der organischen Naturwissenschaften) genannt werden, und wenn Herr Dr. Hahn seine Krinoiden aus den Korallen und diese wieder aus den Schw"ammen "`durch Vermehrung der Kan"ale"' entstehen l"asst, so ist dies ein Vorgang, der mit dem, was wir "uber die Phylo- und Ontogenie der Protozoen, C"olenteraten und Echinodermen wissen, g"anzlich unvereinbar ist. Gerade der auf S. 33 des Hahn'schen Werkes hervorgehobene "`einheitliche"' Typus der meteoritischen Organismen und der Umstand, dass s"amtliche Formen in eine "Ubergangsreihe gestellt werden k"onnen, scheinen mir gewichtige Argumente gegen die organische Natur der Chondren zu bilden. Welcher Zoologe oder Pal"aontologe wird in den Spongien, den Korallen und den Krinoiden einen einheitlichen Typus erblicken?

Die "`Struktur"' der Chondren gemahnt allerdings im großen Ganzen an gewisse R"ohrenkorallen, und kann man auch einem Laien, wenn man nachsichtig sein will, die Verwechslung mit den terrestrischen \emph{Favositen} verzeihen. Manche Chondren zeigen keine Gliederung; diese werden als die primitivsten betrachtet und sowohl von Dr. Hahn als auch von Dr. Weinland f"ur Schw"amme gehalten. Macht sich eine Gliederung in mehr oder weniger radial verlaufende S"aulchen bemerkbar, so entsteht, namentlich wenn auch Querscheidenw"ande vorhanden find (was durchaus nicht immer der Fall ist), eine "`unzweifelhafte"' R"ohrenkoralle. Zieht sich durch die quergegliederten S"aulchen ein zentraler L"angskanal hindurch, so ist der "`unzweifelhafte"' Krinoide fertig. Die Entwicklung geht manchmal so rapid vor sich, dass aus einem Schwamm direkt ein Krinoide entsteht. Ein solches Avancement machte z. B. das von Dr. Hahn auf Tafel 30, Fig. 5 abgebildete Exemplar mit; es ist dies ein "`unzweifelhafter"' Krinoide, der mit allem Stolze eines Parvenu auf die dunklen Tage zur"uckblicken darf, welche er als "`Schwamm"' in der Sammlung des Herrn Dr. Hahn verlebte. G"umbel hat die Struktur der Chondren, die ich als "`Favositoid"' bezeichnen will, mit der Struktur der Hagelk"orner verglichen, ein Vergleich, der in jeder Hinsicht treffend genannt werden kann. Die Excentricit"at des Ausstrahlungspunktes der Fasern ist wohl Regel, doch fand ich im Tieschitzer Meteoriten auch einen Einschluss, in welchem sich die Fasern innerhalb des K"ugelchens treffen. Auch die Beobachtung G"umbels, dass in manchen K"ugelchen (Chondren) "`gleichsam mehrere nach verschiedenen Richtungen hin strahlende Systeme"' vorhanden find und dadurch eine "`scheinbar wirre, st"anglige Struktur"' zum Vorschein kommt, konnte ich mehrfach best"atigen. Die favositoide Struktur der Chondren ist nur eine seiner Ausbildung des auch an anderen Einschl"ussen der Chondrite vorkommenden "`s"auligen"' Baues; den letzten konnte ich an einem Feldspat (?) beobachten, dessen geradlinige Umrisse ziemlich deutlich erkennbar find; die Lamellen, resp. S"aulen, find hier wohl nicht radial angeordnet, jedoch dadurch besonders interessant, dass in der Mitte derselben mehrere in eine L"angsreihe angeordnete runde Glaseinschl"usse sich bemerkbar machen. Solche kleinen Einschl"usse find es offenbar, die f"ur Perforationen gehalten wurden, analog denen, wie sie an den R"ohrenw"anden der \emph{Favositen} vorkommen. Manchmal verschwimmen die einzelnen rundlichen Tr"opfchen zu einem scheinbaren, die Mitte der S"aule durchziehenden Kanale. Die vermeintlichen Wand"offnungen finden sich auch dort, wo keine Querscheidew"ande die "`Korallenr"ohre"' abteilen. Die Querscheidew"ande sehen "uberhaupt sehr oft, und wo sie entwickelt find, geben sie sich durch die Unregelm"aßigkeit und Unbestimmtheit ihres Auftretens als einfache Querkl"ufte zu erkennen, wie ich sie in makroskopischer Ausbildung am Enstatit von Zdjar und an den Turmalins"aulen von Rozna in M"ahren beobachten konnte. Unm"oglich kann man die "`Querscheidew"ande"' der Chondren als wirkliche, durch organische T"atigkeit gebildete und den Dissepimenten der terrestrischen Korallen analoge Querw"ande betrachten. G"umbel, der mit mikropal"aontologischen Untersuchungen vertraut ist, w"urde die organische Struktur der "`zierlich quergegliederten Fasern"' gewiss erkannt haben, wenn man es hier "uberhaupt mit einer solchen zu tun h"atte.

Was nun endlich die "Ahnlichkeit der Chondren mit bekannten Formen anbelangt, so ist dieselbe h"ochstens eine "außerliche. Kann ein Gegenstand, den man zuerst f"ur eine Pflanze, dann f"ur einen Seeschwamm und endlich f"ur einen Krinoiden erkl"art, mit einer "`bekannten Form"' "Ahnlichkeit haben? Ich bin "uberzeugt, dass sich niemand von diesem Proteus eine deutliche Vorstellung zu bilden vermag.

Aus dem Gesagten geht hervor, dass die f"unf von Herrn Dr. Hahn ausgestellten Bedingungen keineswegs den Beweis einer tierischen Natur der Chondren in sich schließen. Wenn (S. 33) die "`Vergesellschaftung gleicher Formen"' als ein "`erhebliches Beweismoment"' f"ur die organische Natur hingestellt wird, so find mit demselben Grade von Wahrscheinlichkeit die Augitkristalle einer Lava oder die H"auser einer Stadt als Organismen zu betrachten. Wie kommt es "ubrigens, dass Herr Dr. Hahn die organische Natur des Eozoon canadense leugnet, obwohl dieses alle die von ihm ausgestellten Bedingungen erf"ullt? Die primitivsten Formen der Chondren erkl"art Dr. Hahn, wie bereits bemerkt, f"ur Schw"amme, und fasst sie unter dem Namen "`\emph{Urania}"' zusammen; er findet eine große Verwandtschaft derselben mit terrestrischen Formen, und erkennt sogar das Genus \emph{Astrospongia} (!). Anwachstellen und Mund"offnungen vermag er an den D"unnschlissen seiner Schw"amme ganz deutlich zu unterscheiden. Undeutliche Gewirre kleiner Kristallleistchen h"alt Herr Dr. Hahn f"ur Nadelger"uste von Spongien; bei einem eventuellen "`Avancement"' einer solchen Nadelspongie zu einem Krinoiden k"onnen die Nadeln nat"urlich unm"oglich Nadeln bleiben, sondern m"ussen nolens volens zu Krinoidenarmen werden. Dass Herr Dr. Hahn solche zoologischen Eskamotagen, die dem enragiertesten Darwinianer das Blut in den Adern erstarren lassen, wirklich durchgef"uhrt hat, davon beliebe man sich auf S. 25 seines Werkes zu "uberzeugen. Jedenfalls wird dadurch die "`Unzweifelhaftigkeit"' der tierischen Natur der Chondren in ein ganz eigent"umliches Licht gestellt.

Was die "`Korallen"' anbelangt, so ist eine Vergleichung oder gar Identifizierung derselben mit terrestrischen Formen nicht zul"assig; da die meisten "`St"ocke"' nur Bruchteile von Millimetern im Durchmesser aufweisen, so kommen den einzelnen "`Polypenr"ohren"' so geringe Dimensionen zu, dass wir kaum berechtigt find anzunehmen, dass diese mikroskopischen St"ocke von Tieren bewohnt waren, die eine n"ahere Verwandtschaft mit den terrestrischen Anthozoen befassen. Aus diesem Grunde hat auch Herr Dr. Weinland die "`\emph{Favositen}"' der Chondrite zu einem neuen Genus, welches er "`\emph{Hahnia}"' nennt, erhoben.

Becher-, R"ohren- und Sternkorallen unterscheiden zu wollen, scheint mir von Herrn Dr. Hahn, abgesehen von allem anderen, doch zu weit gegangen zu sein.

Die Krinoiden finden sich nach Dr. Hahn "`von der einfachsten Form eines gegliederten Armes angefangen bis zum ausgebildeten Krinoiden mit Stiel, Krone, Haupt- und Hilfsarmen"'. Als Krinoiden werden z. B. die Figuren 1 und 2, Tafel 25, angesprochen; sie sehen jedoch gar nicht danach aus, indem die angeblichen Krinoidenarme "uberall gleich breit und durchaus einfach find, w"ahrend sie sich ja doch bekanntlich in Wirklichkeit von der Krone weg zuspitzen und gew"ohnlich auch verzweigen. Die Gliederung der "`Arme"' ist so unregelm"aßig und unvollkommen, dass dadurch niemand, der einen Krinoiden kennt, an einen solchen gemahnt wird. Die "`Knickung"' der Arme l"asst sich nach Dr. Hahns Ansicht nur bei Krinoiden denken; w"are diese Knickung nicht da, so h"atte Herr Dr. Hahn den unzweifelhaften Krinoiden vielleicht f"ur eine ebenso unzweifelhafte — Koralle erkl"art! Nachdem ich einen der bereits erw"ahnten, quergegliederten Enstatitkristalle ebenfalls geknickt finde, muss ich denselben wohl auch als "`Krinoidenarm"' betrachten?

Manche "`Krinoiden"' bestehen nach Herrn Dr. Hahn "`bloß aus einer beliebigen Anzahl von Armen"'; die Stiele und Kronen scheinen also diesen Krinoiden zu fehlen, und finde ich es deshalb vollkommen gerechtfertigt, wenn sie Herr Dr. Hahn als eine "`besondere Art"' hinstellt. Diesen "`besonderen"' Krinoiden k"onnte etwa ein Fisch, der nur aus Flossen besteht, w"urdig zur Seite gestellt werden.

Es d"urfte f"ur Viele von Interesse sein, zu erfahren, dass Herr Dr. Hahn bei vielen seiner Krinoiden nicht nur Stiel und Krone, sondern auch die "`Mund"offnung zwischen den H"ockern"', und — man h"ore und staune — sogar Muskelschichten ganz deutlich beobachtet hat!!

Wenn man die angeblichen Organismen der Chondrite mit terrestrischen Formen vergleicht, so muss man f"ur dieselben wohl auch "ahnliche Existenzbedingungen vorausfetzen; aus dieser Voraussetzung muss man konsequent folgern, dass die Chondrite als ein Analogon der terrestrischen klastischen Gesteine zu betrachten find. Gegen diese mit logischer Notwendigkeit sich ergebende Auffassung spricht sich Herr Dr. Hahn auf das Entschiedenste aus und sucht f"ur die Chondrite eine Bildungsweife geltend zu machen, welche unsere bisherigen Ansichten "uber die Kosmogonie wesentlich alteriert. Geht man indessen von den Pr"amissen, wie sie Herr Dr. Hahn aufgestellt hat, aus, und versucht auf streng logischem Wege die Schl"usse zu ziehen, so ger"at man bald in ein Chaos von Widerspr"uchen, die absolut nicht zu l"osen sind.

Auch vom Standpunkte des Chemikers k"onnte man dem Hahn'schen Werke mancherlei Einwendungen machen; ich will mich indessen darauf nicht weiter einlassen und nur noch erw"ahnen, dass man solche Ansichten, wie sie Herr Dr. Hahn z. B. "uber die Entstehung der Gebirge und "uber die Vulkane entwickelt, heutzutage selbst einem Laien unm"oglich verzeihen kann.

Br"unn, am 25. April 1881.
\end{document}
